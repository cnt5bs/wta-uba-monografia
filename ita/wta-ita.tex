\documentclass[12pt,graybox,envcountchap,sectrefs]{svmono}


%%% Para uso de figuras neste documento.
%%% Deve ser redefinido para gerar a monografia
\newcommand{\itaPastaFig}{fig}
\newcommand{\geralPasta}{../geral}
\newcommand{\itaPasta}{.}

\usepackage{polyglossia}
\setmainlanguage{portuges}

% choose options for [] as required from the list
% in the Reference Guide
%\usepackage{mathptmx}
%\usepackage{helvet}
%\usepackage{courier}
\usepackage{xltxtra,fontspec,xunicode}
%\setmainfont[Ligatures=TeX]{Vollkorn}
%\setromanfont[Ligatures=TeX]{Vollkorn}
%\setsansfont[Ligatures=TeX,Scale=MatchLowercase]{DejaVu Sans}
%\setmonofont[Scale=MatchLowercase]{DejaVu Sans Mono}

\usepackage{amsmath}
\usepackage{amssymb}
\newfontfamily\listingsfont[Scale=0.7]{DejaVu Sans Mono}
\newfontfamily\listingsfontinline[Scale=0.8]{DejaVu Sans Mono}
\usepackage{listings}
	\usepackage{color}
	\definecolor{sh_comment}{rgb}{0.12, 0.38, 0.18 } %adjusted, in Eclipse: {0.25, 0.42, 0.30 } = #3F6A4D
	\definecolor{sh_keyword}{rgb}{0.37, 0.08, 0.25}  % #5F1441
	\definecolor{sh_string}{rgb}{0.06, 0.10, 0.98} % #101AF9
	\def\lstsmallmath{\leavevmode\ifmmode \scriptstyle \else  \fi}
	\def\lstsmallmathend{\leavevmode\ifmmode  \else  \fi}
	\lstset {
	language=Java,
%	frame=shadowbox,
	rulesepcolor=\color{black},
	showspaces=false,showtabs=false,tabsize=2,
%	numberstyle=\tiny,numbers=left,
	basicstyle= \listingsfont,
	stringstyle=\color{sh_string},
	keywordstyle = \color{sh_keyword}\bfseries,
	commentstyle=\color{sh_comment}\itshape,
	captionpos=b,
	xleftmargin=0.7cm, xrightmargin=0.5cm,
	lineskip=-0.3em,
	escapebegin={\lstsmallmath}, escapeend={\lstsmallmathend}
	}
	% Applies only when you use it
	\usepackage{xcolor}
	\lstdefinestyle{antlr}{
	basicstyle=\small\ttfamily\color{magenta},%
	breaklines=true,%                                      allow line breaks
	moredelim=[s][\color{green!50!black}\ttfamily]{'}{'},% single quotes in green
	moredelim=*[s][\color{black}\ttfamily]{options}{\}},%  options in black (until trailing })
	commentstyle={\color{gray}\itshape},%                  gray italics for comments
	morecomment=[l]{//},%                                  define // comment
	emph={%
	 STRING%                                            literal strings listed here
	 },emphstyle={\color{blue}\ttfamily},%              and formatted in blue
	alsoletter={:,|,;},%
	morekeywords={:,|,;},%                                 define the special characters
	keywordstyle={\color{black}},%                         and format them in black
	}
\usepackage{type1cm}
\usepackage{makeidx}         % allows index generation
\usepackage{graphicx}        % standard LaTeX graphics tool
                             % when including figure files
\usepackage{multicol}        % used for the two-column index
\usepackage[bottom]{footmisc}% places footnotes at page bottom

% Bibliography
%\usepackage{natbib,har2nat}
\usepackage[numbers]{natbib}
\renewcommand{\refname}{Bibliografia}

\usepackage[pdfpagelabels=true, plainpages=false,
            colorlinks=true, allcolors=blue]{hyperref}

% see the list of further useful packages
% in the Reference Guide
\makeindex             % used for the subject index
                       % please use the style svind.ist with
                       % your makeindex program

%\usepackage{epigraph}

\newcommand{\fig}[4][0.50]{\begin{figure}[ht]
%\sidecaption
\includegraphics[scale=#1]{#2/#3}
\caption{#4}
\label{#3}       % Give a unique label
\end{figure}}

\newcommand{\figBot}[4][0.50]{\begin{figure}[b]
%\sidecaption
\includegraphics[scale=#1]{#2/#3}
\caption{#4}
\label{#3}       % Give a unique label
\end{figure}}

\newcommand{\figTop}[4][0.50]{\begin{figure}[t]
%\sidecaption
\includegraphics[scale=#1]{#2/#3}
\caption{#4}
\label{#3}       % Give a unique label
\end{figure}}

\newcommand{\figLatHere}[4][0.50]{\begin{figure}[ht]
\sidecaption
\includegraphics[scale=#1]{#2/#3}\caption{#4}\label{#3}
\end{figure}}

\newcommand{\figLatTop}[4][0.50]{\begin{figure}[t]
\sidecaption
\includegraphics[scale=#1]{#2/#3}\caption{#4}\label{#3}
\end{figure}}

\newcommand{\figLat}[4][0.50]{\begin{figure}[b]
\sidecaption
\includegraphics[scale=#1]{#2/#3}\caption{#4}\label{#3}
\end{figure}}

%%% Elementos adaptativos
\newcommand{\naoTerminal}[1]{\lstinline[style=antlr]{#1}}
\newcommand{\terminal}[1]{`\lstinline[style=antlr]{#1}'}
\newcommand{\uml}[1]{\lstinline{#1}}
\newcommand{\codigo}[1]{\lstinline{#1}}
\newcommand{\uba}[1]{\lstinline{#1}}
\newcommand{\instancia}[1]{\lstinline{#1}}
\newcommand{\funcao}[1]{\lstinline{#1}}
\newcommand{\estado}[1]{\lstinline{#1}}
\newcommand{\simbolo}[1]{\lstinline{#1}}
\newcommand{\regra}[1]{``\lstinline{#1}''}
\newcommand{\acao}[1]{``\lstinline{#1}''}
%%%%%%%%%%%%%%%%%%%%%%%%%%%%%%%%%%%%%%%%%%%%%%%%%%%%%%

\newcommand{\bibPasta}{../../projeto-git/bib}


\title{Linguagem LUBA}
\author{Italo S. Vega}

\begin{document}

\maketitle

\tableofcontents

\begin{partbacktext}
\part{Fundamentação Teórica}
%\noindent Use the template \emph{part.tex} together with the Springer document class SVMono (monograph-type books) or SVMult (edited books) to style your part title page and, if desired, a short introductory text (maximum one page) on its verso page in the Springer layout.

\end{partbacktext}


% Este tópico é essencial na obra, e recomendamos que seja elaborado com cuidado. Seu objetivo é o de, através de uma revisão criteriosa dos conceitos científicos fundamentais em que o material do livro se baseia, recordar os assuntos tratados aos leitores que já os conhecem e ao mesmo tempo, orientar conceitualmente os leitores que não possuam conhecimentos prévios dos assuntos pertinentes a cada uma das disciplinas envolvidas nos assuntos tratados, permitindo-lhes uma compreensão, ainda que básica, do tema tratado sem que necessitem acessar outras fontes de referência, tornando assim auto-contido o material apresentado no livro.

% Nele devem ser apresentados, com relação tanto às áreas básicas como às de aplicação, todos os elementos científicos que dão suporte à tecnologia que o livro pretende sugerir. Assim, sugere-se a inclusão de uma seção de definições, na qual se esclareça:

% o significado preciso de toda a terminologia empregada,

% o significado da simbologia utilizada nas notações formais utilizadas

% no caso de assuntos inter ou multidisciplinares, a exata relação entre os elementos das diversas disciplinas envolvidas

% Na composição dos assuntos reunidos sob a classe Fundamentos, subentendem-se elementos conceituais: teorias, bases matemáticas, notações formais, modelos de representação, teoremas, demonstrações, propriedades, relações de correspondência e de equivalência.

% Não menos importantes, devem ser apresentadas, quando couber, analogias com fenômenos naturais, ouabstratos, taiscomocomocomparaçõesgeométricas, algébricas, linguísticas, físicas, biológicas, sociológicas e quaisquer outras consideradas relevantes

% A leitura do tópico Fundamentos deve permitir que o leitor exercite e aprimore seu potencial de absorver os conteúdos dos capítulos tecnológicos subsequentes, ainda que não seja um especialista em qualquer das áreas de conhecimento tratadas na obra.

	%\motto{Use the template \emph{chapter.tex} to style the various elements of your chapter content.}
\chapter{Adaptatividade e Semântica Operacional}
\label{cap-adaptatividade} % Always give a unique label
% use \chaptermark{}
% to alter or adjust the chapter heading in the running head

%\epigraph{The data are accumulating and the computers are humming, what we are lacking are the words, the grammar and the syntax of a new language...}{Dennis Bray (TIBS22(9):325-326,1997)}

% adaptatividade como problema computacional
% adaptatividade sob o ponto de vista biológico
% como estabelecer esta conexão?
% "Isso não cobre, no entanto, as necessidades daqueles usuários cujos programas necessitem efetuar alterações dinâmicas,  durante a execução, na lógica de seus procedimentos.  Para isso, tais programas necessitariam incorporar recursos de adaptatividade  e, portanto, dispor de meios para, sem intervenção externa, determinar e realizar alguma automodificação estrutural necessária." (JJNeto, Um Levantamento da Evolução da  Adaptatividade e da Tecnologia Adaptativa , 2007)
% IDEIA de conexão: o MTS produz uma especificação de modelo que embute a adaptatividade intrinsica ao comportamento biológico. Uma implementação "tradicional" de tal modelo, envolveria uma ação externa para ser implementado. A tecnologia adaptativa oferece recursos para implementar este modelo sem tal restrição.
% link para usar UML como notação de representação de modelos

\abstract{Neste capítulo, apresenta-se uma semântica operacional para autômatos adaptativos com base em regras de transição. Inicialmente, desenvolve-se o modelo semântico para suportar a especificação de instâncias de autômatos subjacentes. Em seguida, extende-se o modelo para acomodar o mecanismo adaptativo. Ao longo da apresentação, ilustra-se a utilização da LUBA em um exemplo de reconhecimento de cadeias de uma linguagem dependente de contexto.}


    



\section{Autômato Subjacente}
\label{sec:isv:subjacente}

%\begin{quote}\textit{(A ideia central desta seção é a de apresentar o modelo semântico de um autômato subjacente por meio de uma notação personalizada. Justifica-se tal notação no sentido de se criar uma linguagem de programação com suporte à implementação dos padrões de desenho adaptativos.)}\end{quote}

Diferentes dispositivos adaptativos podem ser explorados por uma tecnologia adaptativa, como já observado em diversos trabalhos de pesquisa sobre adaptatividade divulgados no endereço  \url{http://lta.poli.usp.br/lta}. Nesta monografia, dispositivos do tipo autômato adaptativo serão utilizados para a apresentação do modelo operacional proposto por Vega~\cite{vega:2008}.
Inicialmente, apresentam-se os conceitos básicos que se referem ao dispositivo subjacente dirigido por regras de transição de estado.

Do ponto de vista conceitual, entende-se que o dispositivo subjacente ao autômato adaptativo, seja constituído por um conjunto de regras que mapeiam um par estado corrente/símbolo de entrada em um novo estado --- o estado seguinte. O diagrama de classes conceitual apresentado na Fig.~\ref{fig-regra-tradicional-uxf}, destaca os importantes elementos desta estrutura, a qual segue uma linha tradicional de formalização. Uma instância da classe \uml{Autômato Subjacente} é constituída por um conjunto de \uml{estados}, um dos quais é dito \uml{inicial}. Também parte desta estrutura é o conjunto dos estados \uml{finais}; um subconjunto de \uml{estados}. Durante a operação do autômato, um dos seus \uml{estados} é conhecido por estado \uml{corrente}. Além de estados, autômatos desta classe também são constituídos por um \uml{Espaço de Regras} --- um conjunto de \uml{regras}, cada uma delas modelada por uma instância de \uml{Regra Subjacente}. Esta parte do modelo de um \uml{Autômato Subjacente} suporta a transição de estados ao longo do reconhecimento de uma cadeia de símbolos~\cite{ramos:2009:lftmi}. Em particular, seja \uml{d} uma instância da classe  \uml{Autômato Subjacente}, com a regra \uml{r = (p, a, q)} $\in$ \uml{regras}. Se o estado \uml{d.corrente} for igual à \uml{r.p} e o estímulo de entrada corresponder à \uml{r.a}, então \uml{r.q} será o novo estado corrente do dispositivo: \uml{d.corrente = r.q}.

	\figLatHere[0.67]{\itaPastaFig}{fig-regra-tradicional-uxf}{Modelo conceitual de um autômato subjacente com destaque para o seu espaço de regras subjacentes.}

Estes elementos conceituais já são suficientes para o projeto inicial da gramática da linguagem de descrição de unidades biomiméticas adaptativas (LUBA). Pelegrini~\cite{pelegrini:2009} apresenta diversos conceitos e tecnologia para o projeto de linguagens que suportam um estilo de programação baseado na noção de adaptatividade. LUBA não é uma linguagem que oferece os recursos necessários para o desenvolvimento de programas em geral. A sua concepção enfatiza apenas os recursos necessários para a exposição das ideias desta monografia: regras de transição e funções adaptativas. A especificação da gramática LUBA será feita com a ferramenta ANTLR4~\cite{parr:2013}, cujos principais elementos podem ser encontrados no Anexo~\ref{anexo:isv:antlr4}.

A primeira regra de produção da gramática LUBA define o não-terminal \naoTerminal{uba} com a seguinte estrutura. No nível léxico, a palavra-reservada \terminal{uba} é seguida pelo identificador da nova classe de autômatos e, entre um par de abre/fecha-chaves, lista-se uma sequência (possivelmente vazia) de declarações de regras de transição. Na linguagem ANTLR4, o não-terminal de cada regra de produção é separado da sua definição por um \terminal{:} e as regras de produção, propriamente ditas, são separadas por um \terminal{;}. Assim, a regra para se especificar uma nova classe de autômatos \terminal{uba} fica:

\begin{lstlisting}[style=antlr]
<<UbaParser::seção de declarações de regras>>=
luba : UBA idClasse ABBL declaracao* FEBL ;
declaracao : decRegra | decFuncao | decMetodo ;
idClasse : IDENTIFICADOR ;
\end{lstlisting}

Posteriormente a linguagem será estendida para acomodar a declaração de funções adaptativas. A especificação de um autômato subjacente não envolve a chamada de funções adaptativas; é suficiente que se especifique qual o par estado corrente/símbolo de entrada dispara a ocorrência da transição para o estado seguinte. Como parte do projeto da linguagem LUBA (e visando facilitar um pouco a escrita das especificações), o estado inicial deverá ser prefixado com \lstinline[style=antlr]'>', enquanto que o prefixo para os estados finais será \lstinline[style=antlr]'!'. Assumindo-se estas convenções, escreve-se a produção do não terminal \lstinline[style=antlr]!decRegra! da seguinte forma:

\begin{lstlisting}[style=antlr]
<<UbaParser::declaração de regra adaptativa>>=
decRegra :
	cfa?
	ABPAR decEstOrg VIRGULA idSimbolo? FEPAR
	SETA decEstDst
	cfp?
	;
decEstOrg : INICIAL? idEstado ;
decEstDst : FINAL? idEstado ;
\end{lstlisting}

\noindent
Observa-se que a produção para o não-terminal \lstinline[style=antlr]!decRegra! admite a escrita de regras de transição em vazio: pode-se omitir o símbolo em uma transição entre dois estados.

Como um primeiro exemplo, a especificação do modelo de um autômato subjacente nos moldes daquele sugerido pelo diagrama mostrado na Fig~\ref{fig-regra-tradicional-uxf} tem a forma:

\begin{lstlisting}
<<exemplo1.uba>>=
uba Exemplo1 {
	( >q1, a ) -> q1
	( q1, b ) -> q2
	( q2, c ) -> !q3
}
\end{lstlisting}

\noindent
A criação de uma instância \uba{a1} da classe \uba{Exemplo1} é escrita da seguinte forma:

\begin{lstlisting}
<<apl1exemplo::instanciação de uma Uba>>=
def d1 = new cenario.Exemplo1()
\end{lstlisting}

\noindent
Na Fig~\ref{fig-cenario-Exemplo1-0-delta.png} encontra-se uma representação visual da topologia do autômato imediatamenta após a sua instanciação. O estado inicial é \estado{q1} (indicado por uma seta de entrada de estado) e, a partir dele, encontra-se uma auto-transição rotulada com o símbolo \simbolo{a}. Além dela, outra transição conduz ao estado \estado{q2}, um estado interno, na ocorrência de um símbolo \simbolo{b} na cadeia de entrada. O estado \estado{q3} é final e encontra-se conectado ao estado \estado{q2} por uma transição que depende do simbolo \simbolo{c} para ocorrer. Neste mesmo diagrama, o estado \estado{q1} encontra-se destacado: ele é o estado corrente do autômato recém instanciado. Instâncias do tipo \lstinline!Exemplo1! reconhecem as cadeias da linguagem denotada pela expressão regular $\mathbf{a^*bc}$, considerando-se o alfabeto $\Sigma=\{a, b, c\}$.

	\figLatHere[0.35]{\itaPastaFig}{fig-cenario-Exemplo1-0-delta.png}{Diagrama de estados de uma instância de \lstinline{Exemplo1} antes de entrar em operação. Em particular, o estado \estado{q1} é inicial e corrente, enquanto o estado \estado{q2} é interno e \estado{q3}, final.}

Do ponto de vista do processador da linguagem LUBA (a ser detalhado no Cap~\ref{cap-isv-execucao}), as regras de produção apresentadas até este momento fazem-no operar de modo que os seguintes efeitos sejam produzidos. Na parte correspondente à declaração da regra de transição não adaptativa \lstinline'( >q1, a ) -> !q1', reconhece-se a introdução de um novo estado \lstinline|q1| na forma \lstinline|>q1| (por ser a primeira ocorrência deste identificador em uma posição de \textit{estado} da regra de transição). O operador \lstinline|>| torna-o inicial na instância \lstinline|a1|. Assim, três efeitos devem ser produzidos:

   \begin{enumerate}

   \item declaração de um novo estado identificado por \lstinline|q1|;

   \item indicação que se trata do estado inicial de \lstinline|a1|;

   \item indicação que se trata do primeiro estado corrente quando tem início o processo de reconhecimento de uma cadeia de símbolos.

   \end{enumerate}

Nesta mesma regra de transição, observamos a forma \lstinline|!q1|. O operador \lstinline|!| torna o estado identificado por \lstinline|q1| um dos estados finais de \lstinline|a1|. Como não é a primeira ocorrência de \lstinline|q1| na descrição \lstinline!Exemplo1!, então não ocorrerá a declaração de um novo estado na instância \lstinline!a1!: aquele anteriormente vinculado a este identificador, além de ser inicial, também será um dos estados finais de \lstinline|a1|.

Agora, qual significado terá à ocorrência do símbolo \lstinline|a|? Corresponde a algum símbolo da cadeia de entrada a ser processado pelo autômato \lstinline|a1|. Ou seja, se o estado corrente de \lstinline|a1| corresponder ao identificador \lstinline|q1| e o símbolo a ser processado corresponder ao símbolo \lstinline|a|, então o autômato \lstinline|d1| passará para o estado identificado por \lstinline|q1|. Em uma notação inspirada na proposta apresentada por Neto e Bravo~\cite{neto:bravo:2002} e Dizeró~\cite{dizero:2010}, denota-se a classe de autômatos adaptativos \lstinline|Exemplo1| por $\mathit{Exemplo1}=(Q, \Sigma, q_0, F, \delta)$ sendo:

	\begin{itemize}
	\item $Q$ um conjunto finito de estados;
	\item $\Sigma$ um conjunto finito de estímulos de entrada;
	\item $q_0 \in Q$ o estado inicial;
	\item $F\subseteq Q$ o conjunto de estados finais;
	\item $\delta:Q\times\Sigma \rightarrow Q$ a função de transição de estados;
	\end{itemize}

\noindent e

	\begin{align*}
	Q = \{q_1, q_2, q_3\},\ \Sigma = \{a\},\ q_0 = q_1,\ F = \{ q_3\} \\
	\\
	\delta \triangleq \{ (q_1, a) \mapsto q_1, (q_1, b) \mapsto q_2, (q_2, c) \mapsto q_3 \} \tag{i}\label{d1-transicao}
	\end{align*}

\noindent
No caso do $\mathit{Exemplo1}$, a função de transição $\delta$ contém três pares de elementos, cada par correspondendo a uma regra de transição. A linguagem reconhecida por autômatos da classe \lstinline!Exemplo1! contém todas as cadeias com a forma especificada pela expressão $\mathbf{a^*bc}$.

A proxima seção introduz o suporte à descrição de funções adaptativas na linguagem LUBA.



    \section{Regras Adaptativas}
\label{sec:isv:regra-adaptativa}

%\begin{quote}\textit{(A ideia central desta seção é a de apresentar um modelo semântico de dispositivo adaptativo usando uma notação personalizada descrita na seção anterior. Mais adiante, tal notação será utilizada para especificar a semântica das Unidades Biomiméticas Adaptativas (UBA), estudadas no Cap.~\ref{cap:isv:uba}. Como base, considera-se o artigo \textit{An adaptive automata operational semantic} \cite{italo:2009} e \textit{Especificações adaptativas e objetos, uma técnica de design de software a partir de statecharts com métodos adaptativos} \cite{vega:2012}.)}\end{quote}

Como modelo de computação, o autômato adaptativo é equivalente à Máquina de Turing~\cite{neto:1998}. Por conseguinte, o seu modelo semântico pode ser especificado com elementos das linguagens de programação de alto nível. Em particular, dando continuidade aos elementos notacionais anteriormente apresentados, serão incorporados o suporte à chamada de funções adaptativas nas regras de transição, bem como o suporte à descrição dos seus efeitos por meio de ações adaptativas. O diagrama mostrado na Fig.~\ref{fig-regra-adaptativa-uxf} enfatiza a principal relação entre as ações adaptativas e o espaço de regras adaptativas do autômato: modificação. Uma ação adaptativa pode modificar as regras de um autômato durante a sua opearação. É esta capacidade de auto-modificação espontânea de comportamento que caracteriza a adaptatividade.

	\figLatHere[0.67]{\itaPastaFig}{fig-regra-adaptativa-uxf}{Modelo conceitual de um autômato adaptativo com destaque para as suas ações adaptativas.}

Embora ofereça um suporte reduzido à adaptatividade, a linguagem LUBA suporta a realização de experimentos que ajudam a apresentar os conceitos utilizados na concepção de modelos biomiméticos. Considere-se a especificação de um novo tipo de autômato contendo uma única regra adaptativa inicialmente:

\begin{lstlisting}
<<exemplo2.uba>>=
uba Exemplo2 {
    fx => (>q1, a) -> q1
    (q1, ) -> q2
    (q2, ) -> !q3
    <<Exemplo2::especificação da função adaptativa>>
}
\end{lstlisting}

\noindent
A instanciação de um autômato desta classe é feita nos mesmos moldes da classe \lstinline|Exemplo1| anteriormente discutida:

\begin{lstlisting}
<<apl2exemplo::instanciação de uma Uba>>=
def d2 = new demo.Exemplo2()
\end{lstlisting}

\noindent
A Fig~\ref{fig-cenario-Exemplo2-0-delta} contém uma representação visual da topologia deste autômato após ser instanciado. O seu estado inicial é \estado{q1} (ou seja, é o estado corrente imediatamente após a instanciação do autômato). A partir deste estado, duas regras podem ser disparadas para levar o autômato para os estados \estado{q1} e \estado{q2} (este último, um estado interno). Uma terceira regra contém uma transição que, ao ser disparada, leva \estado{a2} do estado \estado{q2} para o estado \estado{q3}. Tanto esta quanto a segunda regra contém o que se chama de \textit{transição em vazio} ou \textit{transição}-$\epsilon$. A primeira regra, entretanto, traz uma novidade: uma chamada da função adaptativa \funcao{fx}. Por este motivo, esta é uma regra adaptativa.

	\figLatHere[0.35]{\itaPastaFig}{fig-cenario-Exemplo2-0-delta}{Configuração inicial do dispositivo \uba{a2} na qual a auto-transição provoca uma chamada da função adaptativa \funcao{fx}, caso o símbolo de entrada seja uma ocorrência de \simbolo{a}. O estado \estado{q1} é inicial e corrente, enquanto o estado \estado{q2} é interno e \estado{q3}, final.}

Como se comporta o autômato \uba{a2} durante o processamento de uma cadeia de símbolos? Certamente ele reconhece a cadeia vazia denotada por $\epsilon$. Ele reconhece a cadeia $a$? A resposta irá depender do efeito provocado pela chamada da função adaptativa \funcao{fx}. A ideia é especificar o autômato adaptativo para que ele seja capaz de reconhecer a linguagem contendo todas as cadeias com a forma da expressão regular $\mathbf{a^nb^{2n}c^{3n}}$, para $n \geq 0$ (exemplo inspirado em uma apresentação feita por Neto~\cite{neto:2008}). Como anteriormente feito para o \lstinline|Exemplo1|, denota-se a classe de dispositivos adaptativos \lstinline|Exemplo2| por $\textit{Exemplo2}=(Q, \Sigma, q_0, F, \delta, \textit{CA})$ onde:

\begin{itemize}
\item $Q$ é um conjunto finito de estados;
\item $\Sigma$ é um conjunto finito de estímulos de entrada;
\item $q_0 \in Q$ é o estado inicial;
\item $F\subseteq Q$ é o conjunto de estados finais;
\item $\delta:Q\times\Sigma \rightarrow Q$ é a função de transição de estados;
\item \textit{CA} é a camada adaptativa de \textit{Exemplo2};
\end{itemize}

\noindent e

	\begin{align*}
	Q = \{q_1, q_2, q_3\},\ \Sigma = \{a\},\ q_0 = q_1,\ F = \{q_3\} \\
	\\
	\delta \triangleq \{ \mathsf{fx}\cdot(q_1, a) \mapsto q_1,  (q_1, \epsilon) \mapsto q_2,  (q_2, \epsilon) \mapsto q_3 \} \tag{i}\label{d2-transicoes}
	\end{align*}

\noindent
Como no exemplo \uba{Exemplo1}, a função de transição  $\delta$ contém três pares, mas o primeiro deles introduz uma chamada da função adaptativa \funcao{fx}. Agora, qual deve ser a definição desta função de modo que \uba{a2} reconheça as cadeias denotadas pelas expressão $\mathbf{a^nb^{2n}c^{3n}}$? Um novo mecanismo se faz presente na linguagem LUBA: definição de funções adaptativas.




    \section{Funções Adaptativas}
\label{sec:isv:chamada-funcao-adaptativa}

Chamadas de funções adaptativas podem afetar a topologia do espaço de regras do autômato durante o processo de reconhecimento de uma cadeia de símbolos de entrada. Para suportar a declaração de funções adaptativas, a linguagem LUBA deverá oferecer recursos para a especificação de ações de consulta, de remoção e de inserção. Na gramática LUBA, a produção do não-terminal \naoTerminal{declaracao} se altera com a nova alternativa \naoTerminal{decFuncao}:

\begin{lstlisting}[style=antlr]
<<UbaParser::declaração de função adaptativa>>=
decFuncao : idFuncao
	ABBL
	seqConsultas?
	seqRemocoes?
	seqInsercoes?
	FEBL
	;
\end{lstlisting}

\noindent
Por conseguinte, a declaração de uma função adaptativa tem início com a sua identificação seguida de um bloco de sequências de ações adaptativas, delimitado por um par de marcadores (no nível léxico, tratam-se doa sinal de abre/fecha-chaves, respectivamente). Dentro do bloco da função, opcionalmente listam-se as sequências de consultas, remoções e inserções. A sequência de consultas tem a forma estabelecida pelo não-terminal \naoTerminal{seqConsultas}:

\begin{lstlisting}[style=antlr]
<<UbaParser::ações de consulta>>=
seqConsultas : CONSULTAS acaoConsulta*;
    acaoConsulta : BARRA cfa? padraoRegra cfp? BARRA ;
\end{lstlisting}

No nível léxico, a palavra-reservada \terminal{consultas} marca o começo de uma sequência de ações de consulta, denotadas por um padrão de regra (no estilo de expressões regulares). Um formato semelhante se aplica às ações de remoção e de inserção, alterando-se apenas a palavra-reservada. Por exemplo, a seguinte declaração de ações adaptativas na função \funcao{fx} complementa a classe de dispositivos \uba{Exemplo2}:

\begin{lstlisting}
<<Exemplo2::especificação da função adaptativa>>=
fx {
    consultas
    / ( @x, )-> q2 /
    / ( q2, )-> @y /
    remoções
    / ( @x, ) -> q2 /
    / ( q2, ) -> @y /
    inserções
    / ( @x, b ) -> ^0 /
    / ( ^0, b ) -> ^1 /
    / ( ^1,   ) -> q2 /
    / ( q2,   ) -> ^2 /
    / ( ^2, c ) -> ^3 /
    / ( ^3, c ) -> ^4 /
    / ( ^4, c ) -> @y /
}
\end{lstlisting}

Mas qual é o efeito da realização das ações adaptativas que constituem o corpo da função \funcao{fx}?

\subsection*{Ações de Consulta}

Ações de consulta fazem uso de padrões para identificar regras de transição que exibem determinadas propriedades durante a operação do autômato. O formato típico do padrão de uma regra em ações de consulta é, essencialmente, uma par constituído pela configuração do autômato em um determinado instante de operação (estado corrente/símbolo de entrada) e o correspondente estado seguinte. O padrão de símbolo de entrada (opcional no padrão de configuração) consiste apenas do seu identificador:

\begin{lstlisting}[style=antlr]
<<UbaParser::padrão de regra>>=
padraoRegra
    : padraoTransicao
    ;
padraoTransicao
    : padraoConfig SETA padraoDestino
    ;
padraoConfig
    : ABPAR padraoOrigem VIRGULA padraoEstimulo? FEPAR
    ;
padraoDestino
    : padraoEstado | padraoTransicao
    ;
padraoOrigem
    : padraoEstado
    ;
padraoEstado
    : estadoLocal | estadoGerado | estadoInstancia
    ;
\end{lstlisting}

\noindent
Já os padrões do estado de origem (corrente) e de destino (seguinte) podem ser expressos ou como uma variável local ou como um dos estados já existentes na instância ou, ainda, como um novo estado (esta situação será discutida na Seção~\ref{sec:acao:insercao}). Um padrão envolvendo estados da instância estabelece um referencial para identificar grupos de regras a partir dos estados locais, prefixados pelo símbolo \terminal{@}. Por exemplo, o que acontece quando o autômato \uba{a2} realiza a ação \acao{/(@x,)->q2/} no contexto de uma consulta? Esta forma de ação adaptativa vincula o identificador \estado{x} a uma variável local à função e cujo valor depende da existência de alguma regra adaptativa que combine com a forma da ação. Neste caso, a regra adaptativa \regra{fx=>(>q1,a)->q1} deverá ser considerada para que se determine o valor de \estado{x}: igual à \estado{q1}. Usando um raciocínio similar, o valor da variável identificada por \estado{y} será \estado{q3}. O efeito da execução de operações de consulta é a vinculação de variáveis de estado locais à identificadores com subsequente valoração por combinação com as regras adaptativas de um autômato.

Uma vez estabelecidos os valores da variáveis locais \estado{x} e \estado{y}, como eles serão utilizados?

\subsection*{Ação de remoção}

As ações de consulta identificam um subconjunto das regras do autômato durante um momento de operação. É sobre este subconjunto que as ações de remoção atuam. Os seus efeitos consistem da eliminação de regras de acordo com um padrão que segue o mesmo formato daquelas de consulta:

\begin{lstlisting}[style=antlr]
<<UbaParser::ações de remoção>>=
seqRemocoes
    : REMOCOES acaoRemocao*
    ;
acaoRemocao
    : BARRA cfa? padraoRegra cfp? BARRA
    ;
\end{lstlisting}

\noindent
No nível léxico, as ações de remoção encontram-se imediatamente após o terminal \terminal{remoções}. Desta maneira, no contexto das ações de remoção, a forma da ação \acao{/(@x,)->q2/} refere-se a um padrão de regra a ser removida do espaço de regras do autômato, considerando que o valor de \estado{x} é igual a \estado{q1} (vinculação produzida pelas ações de consulta). Considerando a tabela de símbolos construída pelos operadores de consulta, as remoções se encarregam de eliminar regras de transições de uma instância de autômatos da classe \uba{Exemplo2}. Uma estratégia de combinação de padrões e de regras mais uma vez é utilizada, ocorrendo a remoção da transição existente entre os estados \estado{q1} e \estado{q2}. A modificação topológica do dispositivo \instancia{a2} pode ser visualizada de acordo com o diagrama mostrado na Fig.~\ref{fig-cenario-Exemplo2-0-delta-a1-manual}.

   \figLatHere[0.3]{\itaPastaFig}{fig-cenario-Exemplo2-0-delta-a1-manual}{Configuração do dispositivo \lstinline|a2| após a remoção da transição $(q_1,\epsilon)\mapsto q_2$. Em destaque, o estado corrente do autômato.}

A execução da próxima ação de remoção, elimina a transição entre os estados \estado{q2} e \estado{q3}, como ilustrado na Fig.~\ref{fig-cenario-Exemplo2-0-delta-a2-manual}. Removidas as transições desejadas do autômato durante a sua operação, novas regras ainda podem ser introduzidas. É este o efeito que se espera das ações de inserção.

   \figLatHere[0.3]{\itaPastaFig}{fig-cenario-Exemplo2-0-delta-a2-manual}{Configuração do dispositivo \lstinline|a2| após a remoção da transição $(q_2,\epsilon)\mapsto q_3$.}

\subsection*{Ação de inserção}
\label{sec:acao:insercao}

\begin{lstlisting}[style=antlr]
<<UbaParser::ações de inserção>>=
seqInsercoes
    : INSERCOES acaoInsercao *
    ;
acaoInsercao
    : BARRA cfa? padraoRegra cfp? BARRA
    ;
\end{lstlisting}

Finalmente, a ação adaptativa \acao{/(@x,b)->^1/} é realizada no contexto de inserção. O efeito de ações de inserção é o de adicionar regras de transição no espaço de regras do autômato. Um dos recursos oferecidos para a realização de tais ações é o mecanismo de geração de identificadores, denotado pelo símbolo \lstinline|^|. Por exemplo, a execução da primeira ação de inserção faz o mecanismo de geração adicionar um novo estado no autômato \instancia{a2} automaticamente. Para efeitos de legibilidade, o novo estado é gerado com o prefixo \estado{g}. Assim, quando se completa esta ação de inserção, a instância \instancia{a2} modifica-se para uma forma como aquela graficamente representada na Fig.~\ref{fig-cenario-Exemplo2-1-cfa-a3-manual}.

   \figLatHere[0.3]{\itaPastaFig}{fig-cenario-Exemplo2-1-cfa-a3-manual}{Configuração do autômato \instancia{a2} após a inserção da transição $(q_1,b)\mapsto g_4$.}

   \figLatHere[0.3]{\itaPastaFig}{fig-cenario-Exemplo2-1-cfa-a4-manual}{Configuração do autômato \instancia{a2} após a inserção da transição $(g_4,b)\mapsto g_5$.}

A inserção de duas novas regras de transição são ilustradas a seguir. Na Fig~\ref{fig-cenario-Exemplo2-1-cfa-a4-manual} observa-se que, após inserir um novo estado \estado{g5} no autômato, também adiciona-se uma transição conectando com o par (\estado{q1}, \simbolo{b}). A Fig~\ref{fig-cenario-Exemplo2-1-cfa-a5-manual} apresenta o resultado da inserção de uma transição em vazio envolvendo os estados \estado{g5} e \estado{q2}.

   \figLatHere[0.3]{\itaPastaFig}{fig-cenario-Exemplo2-1-cfa-a5-manual}{Configuração do autômato \instancia{a2} após a inserção da transição $(g_5,\epsilon)\mapsto q_2$.}

Ao término da execução de todas as ações de inserção especificadas na função \funcao{fx}, o autômato passa a exibir a configuração ilustrada na Fig.~\ref{fig-cenario-Exemplo2-1-cfp-manual}. Ao processar o símbolo \simbolo{a}, cinco novos estados foram introduzidos na instância \uba{a2}. Além disso, ele adquiriu a capacidade de reconhecer duas ocorrências consecutivas do símbolo \simbolo{b} e três do símbolo \simbolo{c}.

   \figLatHere[0.3]{\itaPastaFig}{fig-cenario-Exemplo2-1-cfp-manual}{Configuração do autômato \uba{a2} após a execução de todas as ações adaptativas de inserção especificadas na função \lstinline|fx| ao reconhecer um símbolo \lstinline!a!.}

Encerradas as alterações especificadas pela função adaptativa \funcao{fx}, realiza-se a transição de estado considerando aquele corrente e o símbolo a ser processado que habilitaram a chamada de \funcao{fx}. Se houver uma chamada para outra função adaptativa (\textit{after}), uma vez mais a topologia do autômato poderá ser alterada. Por exemplo, ao processar o segundo símbolo $a$ da cadeia $aa$, o autômato adquire uma estrutura de estados e de transições com a forma mostrada na Fig.~\ref{fig-cenario-Exemplo2-3-cfp-manual}. Esta topologia do autômato não mais se alterará, uma vez que o sufixo da cadeia de símbolos de entrada é $b^4c^6$ e não contém o símbolo $a$. Qual a conclusão? Partindo-se de um autômato com a configuração inicial mostrada na Fig.~\ref{fig-cenario-Exemplo2-0-delta}, após sucessivas adaptações desencadeadas pelo padrão de símbolos de entrada, ele adquire a capacidade de reconhecer cadeias da linguagem descrita pela expressão $\mathbf{a^nb^{2n}c^{3n}}$. Este é um exemplo que demonstra o conceito de adaptatividade.

   \figTop[0.3]{\itaPastaFig}{fig-cenario-Exemplo2-3-cfp-manual}{Configuração do autômato \lstinline|a2| após a execução de todas as ações adaptativas de inserção especificadas na função \funcao{fx} ao reconhecer dois símbolos \lstinline!a!.}

Isto conclui a apresentação do conceito de adaptatividade segundo a óptica da semântica operacional. A seguir, serão detalhados o projeto da linguagem LUBA e do modelo de execução dos autômatos por ela especificados.


	%%% existem várias técnicas computacionais
	%%% escolha recai sobre adaptatividade
	%%% \input{edu/capAreaUso}
	%%% \input{ita/capAreaUso}

\begin{partbacktext}
\part{Linguagem para Descrição de UBAs}
\end{partbacktext}

	\chapter{Geração de Autômatos UBA}
\label{cap-geracao}

\abstract{Neste capítulo, o modelo semântico anteriormente apresentado serve de base para o projeto da linguagem de especificação de unidades biomiméticas (LUBA). Ao longo da discussão, ilustra-se a utilização da LUBA em um exemplo de reconhecimento de cadeias de uma linguagem dependente de contexto.}

    \section{Projeto da LUBA}
\label{sec:isv:antlr}

Como já apontado por Pelegrini~\cite{pelegrini:2009}, "uma linguagem não é capaz de se auto-modificar", o que poderia ser sugerido pela expressão linguagem adaptativa. Por este motivo, a linguagem LUBA não é uma linguagem adaptativa; é uma linguagem para programação adaptativa, especificamente projetada para a apresentação de conceitos de modelagem de software biologicamente inspirado.

Por outro lado, a implementação de uma linguagem deve ser feita por meio da construção de um programa que reaja, de forma apropriada, às diferentes cadeias que lhe são fornecidas como entrada. Uma categoria de tais programas é conhecida por tradutor, pois converte cadeias de uma linguagem para outra. Neste trabalho, especificações de autômatos serão escritas na linguagem LUBA e traduzidas para a linguagem Groovy.

E como o programa de tradução funciona? Em primeiro lugar, ele deve ser capaz de reconhecer cada trecho de uma especificação de autômato. Por exemplo, a entrada \simbolo{(q1,a)->q2} deve ser reconhecida com uma regra de transição não adaptativa que leva o autômato da configuração \simbolo{(q1,a)} para o estado \estado{q2}. Concluído o reconhecimento, o tradutor deverá realizar uma operação apropriada, como \codigo{traduzirRegra("q1", "a", "q2")}.

A parte de um tradutor que reconhece linguagens é chamada de analisador sintático. Sintaxe se refere à regras que governam a relação de pertinência de uma linguagem. Utilizam-se gramáticas para especificar a sintaxe de uma linguagem. Uma gramática é um conjunto de regras, cada uma delas expressando a estrutura da cadeia de símbolos de entrada. Diferentes ferramentas para ajudar na programação de analisadores sintáticos a partir de gramáticas foram produzidas ao longo da história e ANTLR é uma delas~\cite{parr:2013}. Em linhas gerais, a ferramenta ANTLR4 produz um analisador sintático a partir de uma gramática escrita na linguagem ANTLR. No caso da LUBA, o processo de utilização da ferramenta ANTLR4 segue o modelo representado no diagrama de atividades UML da Fig~\ref{fig-antlr-1-dat}. Retângulos denotam os artefatos de entrada e de saída da análise sintática (denotada por cantos arredondados). As setas representam fluxos de entrada e de saída dos artefatos.

   \figLatHere[0.6]{\itaPastaFig}{fig-antlr-1-dat}{Diagrama de atividades UML da geração do analisador sintático a partir da gramática LUBA. O processo de análise sintática é realizado pela ferramenta ANTLR4.}

Mais especificamente, quais são os elementos sintáticos da linguagem LUBA? É isso que será tratado na próxima seção, seguindo as diretrizes de projeto propostas por Parr~\cite{parr:2007}.

    \section{Gramática LUBA}
\label{sec:isv:gramatica}

A especificação ANTLR da gramática da linguagem LUBA inicia com a definição do não-terminal \naoTerminal{luba}:

\begin{lstlisting}[style=antlr]
luba : UBA idClasse ABBL declaracao* FEBL ;
\end{lstlisting}

\noindent
Uma especificação de autômato \terminal{uba} introduz o nome de uma nova classe de autômatos. Em seguida, insere-se uma sequência de declarações no seu escopo (no nível léxico, um par de abre/fecha-chaves delimita a região de escopo), sejam elas declarações de regras de transição, sejam elas declarações de funções adaptativas:

\begin{lstlisting}[style=antlr]
declaracao : decRegra | decFuncao ;
\end{lstlisting}

\noindent
E como deve ser feita a declaração de uma regra de transição?

\subsection*{Declaração de Regras de Transição}

Uma declaração de regra de transição deve suportar a especificação do movimento de um estado de origem para um estado de destino, eventualmente no contexto da presença de um particular símbolo de entrada. No caso de uma regra adaptativa, pode ocorrer a chamada de uma função anterior e posterior, que serão tratadas na seção seguinte:

\begin{lstlisting}[style=antlr]
decRegra :
	cfa?
	ABPAR decEstOrg VIRGULA idSimbolo? FEPAR SETA decEstDst
	cfp?
	;
\end{lstlisting}

\noindent
No nível léxico, um par de abre/fecha-parênteses estabelece a configuração de transição de estado de um autômato. Tanto o estado corrente (indicado como origem da transição) quanto o estado seguinte (indicado como estado de destino da transição) são prefixados por um indicador de inicial ou pertencer ao conjunto de estados finais, respectivamente:

\begin{lstlisting}[style=antlr]
decEstOrg : INICIAL? idEstado ;
decEstDst : FINAL? idEstado ;
\end{lstlisting}

\noindent
Tanto o identificador de estado quanto o identificador de símbolos e de funções adaptativas seguem o formato tradicional de identificadores: uma letra seguida de diversas letras e dígitos (\terminal{IDENTIFICADOR}):

\begin{lstlisting}[style=antlr]
idEstado : IDENTIFICADOR ;
idSimbolo : IDENTIFICADOR ;
idFuncao : IDENTIFICADOR ;
\end{lstlisting}

Finalmente, as chamadas de funções adaptativas. Na declaração de uma regra de transição, elas aparedem apenas como identificadores:

\begin{lstlisting}[style=antlr]
cfa : idFuncao FANT ;
cfp : FPOS idFuncao ;
\end{lstlisting}

\noindent
O corpo de uma função adaptativa é constituído por uma sequência de ações adaptativas, o que será tratado a seguir.

\subsection*{Declaração de Funções Adaptativas}

A declaração de uma função adaptativa assume um importante papel nos modelos biomiméticos, uma vez que os seus efeitos é que caracterizam a presença da adaptatidade nos comportamentos que podem ser observados. A declaração deve identificar a função e explicitar as sequências de ações de consulta, de remoção e de inserção:

\begin{lstlisting}[style=antlr]
decFuncao : idFuncao
	ABBL seqConsultas? seqRemocoes? seqInsercoes? FEBL ;
seqConsultas : CONSULTAS acaoConsulta*;
seqRemocoes : REMOCOES acaoRemocao*;
seqInsercoes : INSERCOES acaoInsercao*;
\end{lstlisting}

A forma sintática de cada ação adaptativa, por sua vez, segue o mesmo formato. As diferenças encontram-se no modelo semântico, que será detalhado na Seção~\ref{sec:isv:geracao}:

\begin{lstlisting}[style=antlr]
acaoConsulta : BARRA padraoRegra BARRA ;
acaoRemocao : BARRA padraoRegra BARRA ;
acaoInsercao : BARRA padraoRegra BARRA ;
\end{lstlisting}

Nas diferentes ações adaptativas, utilizam-se dois padrões: um que se refere ao padrão de configuração do autômato em operação e outro que se refere ao padrão de destino da transição:

\begin{lstlisting}[style=antlr]
padraoRegra : padraoConfig SETA padraoDestino ;
\end{lstlisting}

O padrão de configuração faz referência ao estado corrente (\naoTerminal{padraoOrigem}) e ao opcional símbolo a ser processado pelo autômato:

\begin{lstlisting}[style=antlr]
padraoConfig
	: ABPAR padraoOrigem VIRGULA padraoSimbolo? 	FEPAR ;
\end{lstlisting}

Finalmente, tanto o padrão de origem quanto o de destino podem ser expressos como um estado de instância, local (prefixado por \terminal{@}) ou como um novo estado (prefixado por \terminal{^}):

\begin{lstlisting}[style=antlr]
padraoOrigem : padraoEstado ;
padraoDestino : padraoEstado ;
padraoEstado 	:
					estadoInstancia
					| estadoLocal
					| estadoGerado ;
\end{lstlisting}

Esta gramática, expressa na linguagem ANTLR4, dirige a geração dos autômatos UBA, como será visto na próxima seção.

    \section{Geração de Autômatos UBA}
\label{sec:isv:geracao}

Embora respeitando o modelo formal de um autômato adaptativo, o projeto do analisador semântico de especificações LUBA envolve uma etapa de natureza criativa, artística. Silva reforça esta posição, enfatizando a dificuldade de se sistematizar o projeto desta etapa dos programas de tradução:

   \begin{quote}
   ``Devido a não existir um padrão para representar a semântica estática de uma linguagem, bem como pela quantidade e tipos de análise variarem de uma linguagem para outra, implementar a análise semântica torna-se uma tarefa não muito bem definida''~\cite{silva:2011}.
   \end{quote}

Com a ajuda da ferramenta ANTLR4, torna-se possível mecanizar a escrita do código que implementa a análise sintática de especificações LUBA. O diagrama UML apresentado na Fig~\ref{fig-antlr-2-dat} revela como este código se integra na estrutura do programa de geração de UBAs. A linha com um losango conectando a classe \uml{Analisador Sintático LUBA} com a classe \uml{Analisador Semântico LUBA} representa a ideia de constituição: o analisador sintático é constituido, estruturalmente, por um analisador semântico. Da mesma forma interpreta-se a conexão entre a classe \uml{Gerador UBA} e o analisador sintático. Ou seja, a classe dos geradores de UBA absorve o analisador sintático na sua estrutura, originando o programa que traduz especificações de autômatos LUBA em máquinas virtuais UBA.

      \figTop[0.6]{\itaPastaFig}{fig-antlr-2-dat}{Diagrama UML revelando a estrutura de classes do \uml{Gerador UBA}.}

Quando em operação, o gerador de UBAs traduz especificações LUBA em máquinas virtuais UBA, programadas na linguagem Groovy (Fig~\ref{fig-antlr-3-dat}). É o modelo de geração de máquinas UBA que será detalhado na próxima seção.

      \figLatHere[0.6]{\itaPastaFig}{fig-antlr-3-dat}{Diagrama de atividades UML da geração de uma máquina virtual UBA a partir da especificação LUBA de um autômato. O processo de tradução é realizado pelo \uml{Gerador UBA}.}

\subsection{Análise Semântica de Especificações LUBA}

Retomando a primeira regra de produção da gramática LUBA (apresentada na Seção~\ref{sec:isv:gramatica}), tem-se:

\begin{lstlisting}[style=antlr]
luba : UBA idClasse ABBL declaracao* FEBL ;
\end{lstlisting}

De forma mecânica, a ferramenta ANTLR4 produz o código do analisador sintático. A cada não-terminal da gramática processada, ele gera um par de métodos de entrada e de saída (do não-terminal). O corpo destes métodos é que devem implementar o modelo semântico da linguagem LUBA. Assim, par-e-par com o não-terminal \naoTerminal{luba}, existem os métodos \codigo{enterLuba()} e \codigo{exitLuba()}. Por uma decisão de projeto, apenas o primeiro método será utilizado para instanciar um novo objeto (semântico) da classe \codigo{Uba}:

\begin{lstlisting}
@Override
public void enterLuba(UbaParser.LubaContext ctx) {
   novaUba = new gerador.Uba()
}
def novaUba // variável que referencia a "nova" uba
\end{lstlisting}

   \begin{quote}
   \textit{Observação para a Comissão do WTA: a parte do texto a seguir está INCOMPLETA e contém apenas as ideias e código centrais. O texto precisa ser refinado! Isto se aplica para as seções \ref{sec:isv:geracao-regra}, \ref{sec:isv:geracao-funcao}, \ref{sec:isv:geracao-consulta}, \ref{sec:isv:geracao-remocao} e \ref{sec:isv:geracao-estados}.}
   \end{quote}

\subsection{Geração de Regra Adaptativa}
\label{sec:isv:geracao-regra}

\begin{lstlisting}[style=antlr]
decRegra :
	cfa?
	ABPAR decEstOrg VIRGULA idSimbolo? FEPAR SETA decEstDst
	cfp?
	;
\end{lstlisting}

\begin{lstlisting}
@Override
public void exitDecRegra(UbaParser.DecRegraContext ctx) {
   def cfa = ""
   if (ctx.cfa()) {
        cfa = ctx.cfa().idFuncao().getText()
   }
   def origem = ctx.decEstOrg().idEstado().getText()
   def estimulo = ""
   if (ctx.idSimbolo()) {
        estimulo = ctx.idSimbolo().getText()
   }
   def destino = ctx.decEstDst().idEstado().getText()
   def cfp = ""
   if (ctx.cfp()) {
        cfp = ctx.cfp().idFuncao().getText()
   }
   novaUba.incorporarRegraAdaptativa(cfa, origem, estimulo, destino, cfp)
}
\end{lstlisting}

\subsection{Geração de Função Adaptativa}
\label{sec:isv:geracao-funcao}

Retomando o exemplo apresentado na Seção~\ref{sec:isv:regra-adaptativa} ...

\begin{lstlisting}
uba Exemplo2 {
   fx => (>q1, a) -> !q1
   (q1, ) -> q2
   (q2, ) -> !q3
	<!-- restante da especificação -->
}
\end{lstlisting}

Em seguida, a instrução contém uma chamada da função adaptativa \lstinline|fx|. Chamadas de tais funções serão refinadas na Seção~\ref{sec:isv:chamada-funcao-adaptativa}. Além de vincular os identificadores de estados, a seção de regras da classe de dispositivos adaptativos também introduz as seguintes informações na tabela de transições de \lstinline|a2|:

\begin{center}\begin{tabular}{c c c c c c c}
Identificador   & Tipo  & CFA    & Corrente & Entrada & Seguinte & CFP \\
\hline
\lstinline|t1|	& Transição & fx  & \lstinline|q1| & \lstinline|a| & \lstinline|q1| & \\
\lstinline|t2|	& Transição &   & \lstinline|q1| & & \lstinline|q2| & \\
\lstinline|t3|	& Transição &   & \lstinline|q2| & & \lstinline|q3| & \\
\end{tabular}\end{center}

\begin{lstlisting}[style=antlr]
decFuncao : idFuncao
	ABBL
	seqConsultas?
	seqRemocoes?
	seqInsercoes?
	FEBL
	;
\end{lstlisting}

\begin{lstlisting}
@Override
public void enterDecFuncao(UbaParser.DecFuncaoContext ctx) {
   funcaoAdaptativa = new gerador.FuncaoAdaptativa()
   def funcaoId = ctx.idFuncao().getText()
   funcaoAdaptativa.cabecalho( funcaoId )
}
def funcaoAdaptativa

@Override
public void exitDecFuncao(UbaParser.DecFuncaoContext ctx) {
   funcaoAdaptativa.encerrar()
   novaUba.incorporarFuncao(funcaoAdaptativa)
}
\end{lstlisting}

\subsection{Geração de Ação de Consulta}
\label{sec:isv:geracao-consulta}

Retomando o exemplo apresentado na Seção~\ref{sec:isv:chamada-funcao-adaptativa} ...

Desta maneira, monta-se a tabela de símbolos para a execução da função adaptativa \lstinline|fx|:

  \begin{center}\begin{tabular}{c c c c c }
  Identificador   & Tipo  & Escopo    & Referência \\
  \hline
  \lstinline|@x|	& Estado & LOCAL  & \lstinline|q1| \\
  \lstinline|@y|	& Estado & LOCAL  & \lstinline|q3| \\
  \end{tabular}\end{center}

\begin{lstlisting}[style=antlr]
acaoConsulta : BARRA padraoRegra BARRA ;
\end{lstlisting}

\begin{lstlisting}
@Override
public void enterAcaoConsulta(UbaParser.AcaoConsultaContext ctx) {
   acao = new gerador.AcaoConsulta()
}
def acao
\end{lstlisting}

\subsection{Geração de Ações de Remoção}
\label{sec:isv:geracao-remocao}

Retomando o exemplo apresentado na Seção~\ref{sec:isv:chamada-funcao-adaptativa} ...

Embora não ocorra alteração na tabela de estados da instância \instancia{a2}, a situação final da sua tabela de transições se altera para:

   \begin{center}\begin{tabular}{c c c c c c c}
   Identificador   & Tipo  & CFA    & Corrente & Entrada & Seguinte & CFP \\
   \hline
   \lstinline|t1|	& Transição & fx  & \lstinline|q1| & \lstinline|a| & \lstinline|q1| &
   \end{tabular}\end{center}

\subsection{Geração de Estados}
\label{sec:isv:geracao-estados}

Retomando o exemplo apresentado na Seção~\ref{sec:isv:subjacente} ...

\begin{lstlisting}
uba Exemplo1 {
	( >q1, a ) -> q1
	( q1, b ) -> q2
	( q2, c ) -> !q3
}
\end{lstlisting}

Todas estas observações podem ser representadas em uma tabela de símbolos:

\begin{center}\begin{tabular}{c c c c c}
Identificador   & Tipo  & Escopo    & Inicial & Final \\
\hline
\lstinline|q1|	& Estado & INSTÂNCIA  & S & N \\
\lstinline|q2|	& Estado & INSTÂNCIA  & N & N \\
\lstinline|q3|	& Estado & INSTÂNCIA  & N & S \\
\end{tabular}\end{center}

\begin{quote}\textit{(Por exemplo, considere a seguinte regra da gramática, referente à declaração de um novo estado de dispositivo:)}\end{quote}

\begin{lstlisting}[style=antlr]
estadoInstancia : ( INICIAL | FINAL )? idEstado ;
\end{lstlisting}

\begin{lstlisting}
@Override
public void exitEstadoInstancia(UbaParser.EstadoInstanciaContext ctx) {
   def estadoId = ctx.idEstado().getText();
   novaUba.incorporarEstado(estadoId);
   if (ctx.INICIAL()) {
        novaUba.inicial(estadoId)
   }
   if (ctx.FINAL()) {
        novaUba.terminal(estadoId);
   }
   estadoInstancia = estadoId
}
\end{lstlisting}



	\chapter{Modelo de Execução de Autômatos UBA}
\label{cap-isv-execucao}

\abstract{Neste capítulo, apresenta-se o modelo de execução de uma máquina UBA. Ao longo da discussão, ilustra-se a execução de uma máquina-exemplo, capaz de reconhecer de cadeias de uma linguagem dependente de contexto.}

    
   \begin{quote}
   \textit{Observação para a Comissão do WTA: a parte do texto a seguir está INCOMPLETA e contém apenas as ideias e código centrais. O texto precisa ser refinado! Isto se aplica para as seções \ref{sec:isv:execucao-uba}, \ref{sec:isv:execucao-cfa} e \ref{sec:isv:execucao-transicao}.}
   \end{quote}

\section{Operação de Autômatos UBA}
\label{sec:isv:execucao-uba}

   \figTop[0.6]{\itaPastaFig}{fig-uba-dcl}{Diagrama de classes UML revelando a estrutura de classes do para utilização da máquina virtual UBA da classe \uml{Exemplo1}.}

\begin{lstlisting}
<<Execucao::reconhecimento de cadeias>>=
def reconhecer( uba, cadeia ) {
    def reconheceu = true
    cadeia.split("\\.").each { simb ->
        def habilitadas = calcularTransicoesHabilitadas( uba, simb )
        chamarFuncoesAdaptativas( uba, habilitadas, CFA.ANTERIOR )
        reconheceu = realizarTransicao( uba, simb, reconheceu )
        chamarFuncoesAdaptativas( uba, habilitadas, CFA.POSTERIOR )
        recalcularEstados( uba, simb )
    }
    uba.finais = uba.ramos.Reconhecimento.calcularFinais( uba )
    ( reconheceu && uba.corrente in uba.finais ) ? true : false
}
\end{lstlisting}


\begin{lstlisting}
<<Execucao::condição de habilitação de regra>>=
def calcularTransicoesHabilitadas( uba, simb ) {
    uba.regras.findAll { i, r ->
        r[1][0] == uba.corrente && r[1][1] == simb
    }
}
\end{lstlisting}

\subsection*{Análise Adaptativa de Cadeias de Símbolos}

      \figLatHere[0.3]{\itaPastaFig}{fig-cenario-Exemplo2-0-delta}{Topologia inicial do autômato \uba{a2}. O estado \estado{q1} é inicial, final e corrente no momento.}

O primeiro símbolo \simbolo{a} da cadeia de entrada habilita a regra \regra{fx=>(q1,a)->q1} para o disparo. Como ela contém uma chamada da função adaptativa \funcao{fx}, a transição é precedida por uma transformação na topologia do autômato. Como se pode observar na Fig~\ref{fig-cenario-Exemplo2-1-estados-manual}, novos estados (\estado{g4}, \estado{g5}, \estado{g6}, \estado{g7} e \estado{g8}) e transições são incorporados ao autômato, mas o estado corrente ainda é \estado{q1}. Terminada a chamada da função \funcao{fx}, ocorre a transição para o estado \estado{q1} (e, portanto, não se percebe alteração no estado corrente), com consumo do símbolo \simbolo{a}. Visualmente, não se observa esta alteração na configuração do autômato; pode-se utilizar a mesma imagem daquela mostrada na Fig~\ref{fig-cenario-Exemplo2-1-estados-manual}.

      \figLatHere[0.3]{\itaPastaFig}{fig-cenario-Exemplo2-1-estados-manual}{Topologia do autômato depois da primeira chamada da função adaptativa \funcao{fx}. A transição habilitada (indicada em vermelho) ainda não aconteceu.}

Na presença de um segundo símbolo \simbolo{a}, uma vez mais a transição da regra adaptativa \regra{fx=>(q1,a)->q1} torna-se habilitada para o disparo. Por causa da chamada da função adaptativa \funcao{fx}, o autômato \uba{a2} sofre uma nova transformação topológica (Fig~\ref{fig-cenario-Exemplo2-2-estados-manual}). Observa-se a incorporação de novos estados (\estado{g9}, \estado{g10}, \estado{g11}, \estado{g12} e \estado{g13}), além de novas transições.

      \figTop[0.3]{\itaPastaFig}{fig-cenario-Exemplo2-2-estados-manual}{Topologia do autômato depois da segunda chamada da função adaptativa \funcao{fx}. A transição habilitada (indicada em vermelho) ainda não aconteceu.}

A Fig~\ref{fig-cenario-Exemplo2-6-estados-manual} ilustra a situação do autômato após processar o prefixo $aabbbb$. Uma decisão de projeto do modelo semântico foi tomada neste caso. O fechamento-$\epsilon$ do estado \estado{g10} é igual ao conjunto de estados \{\estado{g10}, \estado{g2}, \estado{g11}\}. A decisão foi a de representar o estado corrente do autômato pelo estado \estado{g10}. Formas alternativas de representação podem ser encontradas na literatura~\cite{ramos:2009:lftmi}.

      \figTop[0.3]{\itaPastaFig}{fig-cenario-Exemplo2-6-estados-manual}{Situação do autômato depois de reconhecer o prefixo $aabbbb$.}

Assim, ao processar o quarto símbolo da cadeia de entrada, $b$, o autômato \uba{a2} passa para o estado \estado{g10}. Considerando-se que o próximo símbolo a ser processado é $c$, o autômato transita daquele estado para o estado \estado{g12}.

      \figTop[0.3]{\itaPastaFig}{fig-cenario-Exemplo2-7-estados-manual}{Situação do autômato depois de reconhecer o prefixo $aabbbbc$. Observe-se que o fechamento-$\epsilon$ do estado \estado{g10} é \{\estado{g10}, \estado{q2}, \estado{g11} \}.}

\section{Chamada de Funções Adaptativas}
\label{sec:isv:execucao-cfa}

\begin{lstlisting}
<<Execucao::chamada de função adaptativa>>=
def chamarFuncoesAdaptativas( uba, habilitadas, k ) {
    habilitadas.each { i->
        try {
            // Pode provocar efeito colateral em 'regras'
            uba."${i.value[k]}"()
        } catch(Exception e) {
        }
    }
}
\end{lstlisting}

\begin{lstlisting}
<<Execucao::recálculo de estados>>=
def recalcularEstados( uba, simb ) {
    uba.estados = []
    uba.regras.each { i, r -> uba.estados += [r[1][0], r[1][2]]}
    uba.estados = uba.estados.unique { a, b -> a <=> b }
}
\end{lstlisting}

\section{Transição de Estado}
\label{sec:isv:execucao-transicao}

\begin{lstlisting}
<<Execucao::transição de estado>>=
def realizarTransicao( uba, simb, reconheceu ) {
    if( reconheceu ) {
        def seguintes = uba.ramos.Reconhecimento.seguintes( uba, uba.corrente, simb )
        if( seguintes ) {
            uba.correnteAnterior = uba.corrente
            uba.corrente = seguintes[ 0 ]
        }
        else reconheceu = false
    }
    reconheceu
}
\end{lstlisting}

\subsection{Transição de Estado Não-Adaptativo}

\begin{lstlisting}
<<Reconhecimento::cálculo das transições seguintes>>=
def static seguintes( uba, corrente, simb ) {
    def fechoCorrente = fechamento( uba, [corrente] )
        def seguintes = []
        fechoCorrente.find { q ->
        uba.regras.find { i, r ->
            if( q == r[1][0] && r[1][1] == simb ) {
                seguintes = fechamento( uba, [r[1][2]] )
                return true
            }
        }
    }
    seguintes
}
\end{lstlisting}

\subsection{Cálculo do Fechamento-$\epsilon$}

Baseado em Ramos et al~\cite{ramos:2009:lftmi}:

\begin{lstlisting}
<<Reconhecimento::cálculo do fechamento transitivo>>=
def static fechamento( uba, p ) {
    def s = p
    def fecho = [:]
    uba.estados.each { e-> fecho[ e ] = 'fora' }
    p.each { e-> fecho[ e ] = 'dentro' }
    while( ! s.isEmpty() ) {
        def u = s.pop()
        uba.regras.each { i, r ->
            if( u == r[1][0] && '' == r[1][1] ) {
                def v = r[1][2]
                if( fecho[ v ] == 'fora' ) {
                    fecho[ v ] = 'dentro'
                    s << v
                }
            }
        }
    }
    fecho*.key.findAll { fecho[it] == 'dentro' }
}
\end{lstlisting}



\subsection{Cálculo dos Estados Finais}

\begin{lstlisting}
<<Reconhecimento::cálculo dos estados finais>>=
def static calcularFinais( uba ) {
    def r = uba.estados.inject( uba.finais ) { c, e ->
        def fecho = fechamento( uba, [e] )
        if( fecho.intersect( c )) c += fecho
        c = c.unique()
    }
    r.sort()
}
\end{lstlisting}



%%%\begin{partbacktext}
\part{Modelagem Biomimética Dirigida por UBAs}
\noindent Use the template \emph{part.tex} together with the Springer document class SVMono (monograph-type books) or SVMult (edited books) to style your part title page and, if desired, a short introductory text (maximum one page) on its verso page in the Springer layout.

\end{partbacktext}


\begin{partbacktext}
\part{Resultados}
\noindent Reflexão a respeito das potenciais aplicações e a criação de um catálogo de UBAs.
\end{partbacktext}

	\chapter{Usos dos Resultados (?)}
	\label{cap:resultados} % Always give a unique label

\input{\geralPasta/C.1.Apendice}
	\chapter{Geração de Autômatos UBA}
\label{cap-geracao}

\abstract{Neste capítulo, o modelo semântico anteriormente apresentado serve de base para o projeto da linguagem de especificação de unidades biomiméticas (LUBA). Ao longo da discussão, ilustra-se a utilização da LUBA em um exemplo de reconhecimento de cadeias de uma linguagem dependente de contexto.}

	
\chapter{Gerador de Máquinas UBA}
\label{cap:isv:semantica} % Always give a unique label
% use \chaptermark{}
% to alter or adjust the chapter heading in the running head

\begin{quote}
\textit{(A ideia deste apêndice é apresentar a implementação completa do gerador de máquinas UBA.}
\end{quote}

	
\chapter{Ferramenta ANTLR4}
\label{anexo:isv:antlr4} % Always give a unique label

\begin{quote}
\textit{(A ideia deste apêndice é mostrar aspectos da instalação da ferramenta ANTLR4 e a sua utilização no contexto deste trabalho.)}
\end{quote}

	\chapter{Notação UML}
\label{cap:isv:uml} % Always give a unique label
% use \chaptermark{}
% to alter or adjust the chapter heading in the running head

\begin{quote}
\textit{(A ideia deste apêndice é apresentar os principais elementos da notação UML utilizados ao longo da monografia.)}
\end{quote}


    % Pasta de bibliografia deve ser relativa à ``dist'':
    \bibliographystyle{\bibPasta/spphys}
	\bibliography{\bibPasta/bibliografia}
	\printindex

\end{document}
