# ExecucaoDot {-}

\begin{lstlisting}[style=java]
<<ExecucaoDot.groovy>>=
package uba.exec

class ExecucaoDot extends Execucao {

    def dot = new uba.dot.DigramaUbaDot()
    def static u = 0
    def prefixo="fig-"
    def pastaFig = "/"

    def nome( uba ) {
        uba.class.name.replaceAll("\\.", "-")
    }

    def reconhecer( uba, cadeia ) {
        uba.iniciarDot()
        dot.gerarDot("$pastaFig$prefixo${nome(uba)}-$u-delta.png", uba, cadeia, false)
        dot.gerarDot( "$pastaFig$prefixo${nome(uba)}-$u-estados.png", uba, cadeia, true )
        u++

        def reconheceu = true
        cadeia.split("\\.").each { simb ->
            def habilitadas = calcularTransicoesHabilitadas( uba, simb )

            chamarFuncoesAdaptativas( uba, habilitadas, CFA.ANTERIOR )

            uba.grupoMarcar()
            recalcularEstados( uba, simb )
            uba.grupoGerar( )
            dot.gerarDot("$pastaFig$prefixo${nome(uba)}-${u}-cfa.png", uba, simb, false)

            reconheceu = realizarTransicao( uba, simb, reconheceu )
            chamarFuncoesAdaptativas( uba, habilitadas, CFA.POSTERIOR )

            uba.grupoMarcar()
            recalcularEstados( uba, simb )
            uba.grupoGerar( )
            dot.gerarDot("$pastaFig$prefixo${nome(uba)}-${u}-cfp.png", uba, simb, false)

            dot.gerarDot( "$pastaFig$prefixo${nome(uba)}-$u-estados.png", uba, simb, true )
            u++
        }
        uba.finais = uba.ramos.Reconhecimento.calcularFinais( uba )
        ( reconheceu && uba.corrente in uba.finais ) ? true : false
    }


    def chamarFuncoesAdaptativas( uba, habilitadas, k ) {
        habilitadas.each { i->
            try {
                // Pode provocar efeito colateral em 'regras'
                uba."${i.value[k]}"()
            } catch(Exception e) {
            }
        }
    }
}

