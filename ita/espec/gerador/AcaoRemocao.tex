# AcaoRemocao {-}

\begin{lstlisting}
<<AcaoRemocao.groovy>>=
package gerador

class AcaoRemocao {

    AcaoRemocao() {
        consultaCondicao = []
    }
    def consultaCondicao
    def consultaVinculacao
    def corpo

    def fechar( fa, cfa='', cfp='' ) {
        def txt = """
            retirar += regras.findAll { i, r ->
            def padrao = r[1].join(\":\")
            padrao.replaceAll(/>|!/, \"\") ==~  /${consultaCondicao.join(':')}/
            }// ação de remoção
        """.stripMargin()
        corpo = [txt]
        //
        //        corpo << "retirar += regras.findAll { i, r ->"
        //        corpo << "r[1].join(\":\") ==~  /${consultaCondicao.join(':')}/"
        //        corpo << "}// ação de remoção \n"
        fa.adicionarRemocao( this )
    }

    def condicaoLocal( estado ) {
        if( estado == "") {
            consultaCondicao << ".*"
        }
        else {
            consultaCondicao << "\$$estado"
        }
    }

    def condicao( estado, primeiro, fim ) {
        consultaCondicao << "$estado"
    }

    def vincular( x, y ) {
    }

    def AcaoRemocao( id, instrucao) {
        super(id, instrucao);
    }

    @Override
    def executar( automato,  tabLocal) {
        def acaoEspecifica = tabLocal.definir(acao);
        def remocaoLst = []
        for (def t : automato.tabTransicoes.tabTransicoes.values()) {
            String l = t.instrucao;
            if (t.instrucao.matches(acaoEspecifica)) {
                remocaoLst.add(t);
            }
        }
        for (def t : remocaoLst) {
            automato.tabTransicoes.remover(t);
        }
        return automato.tabTransicoes;
    }

}

\end{lstlisting}

