%\motto{Use the template \emph{chapter.tex} to style the various elements of your chapter content.}

\chapter{Gramática Concreta da Linguagem LUBA}
\label{cap:isv:gramatica} % Always give a unique label
% use \chaptermark{}
% to alter or adjust the chapter heading in the running head

\begin{quote}
\textit{(A ideia deste apêndice é apresentar a especificação completa da gramática da notação UBA. As regras ainda estão sendo trabalhadadas; a seguir encontram-se aquelas utilizadas para a apresentação da noção de adaptatividade no Cap.~\ref{cap-adaptatividade}.)}
\end{quote}

\begin{lstlisting}[style=antlr]
parser grammar UbaParser;
options { tokenVocab=UbaLexer; }

luba : UBA idClasse ABBL declaracao* FEBL ;
declaracao : decRegra | decFuncao ;
idClasse : IDENTIFICADOR ;

// Declaração de regra
decRegra :
	cfa?
	ABPAR decEstOrg VIRGULA idSimbolo? FEPAR
	SETA decEstDst
	cfp?
	;
decEstOrg : INICIAL? idEstado ;
decEstDst : FINAL? idEstado ;
idSimbolo : IDENTIFICADOR ;
cfa : idFuncao FANT ;
cfp : FPOS idFuncao ;
idFuncao : IDENTIFICADOR ;

// Declaração de função adaptativa
decFuncao : idFuncao
	ABBL seqConsultas? seqRemocoes? seqInsercoes? 	FEBL ;
	
seqConsultas : CONSULTAS acaoConsulta*;
seqRemocoes : REMOCOES acaoRemocao*;
seqInsercoes : INSERCOES acaoInsercao*;

acaoConsulta : BARRA padraoRegra BARRA ;
acaoRemocao : BARRA padraoRegra BARRA ;
acaoInsercao : BARRA padraoRegra BARRA ;

padraoRegra : padraoConfig SETA padraoDestino ;
padraoConfig
	: ABPAR padraoOrigem VIRGULA padraoEstimulo? FEPAR ;
padraoOrigem : padraoEstado ;
padraoDestino : padraoEstado ;
padraoEstado
	: estadoLocal | estadoInstancia | estadoGerado  ;

estadoLocal : GERADORVAR IDENTIFICADOR ;
estadoGerado : GERADORANON NUM ;
estadoInstancia : ( INICIAL | FINAL )? idEstado ;
padraoEstimulo : IDENTIFICADOR ;
idEstado : IDENTIFICADOR ;
\end{lstlisting}
