



\section{Autômato Subjacente}
\label{sec:isv:subjacente}

%\begin{quote}\textit{(A ideia central desta seção é a de apresentar o modelo semântico de um autômato subjacente por meio de uma notação personalizada. Justifica-se tal notação no sentido de se criar uma linguagem de programação com suporte à implementação dos padrões de desenho adaptativos.)}\end{quote}

Diferentes dispositivos adaptativos podem ser explorados por uma tecnologia adaptativa, como já observado em diversos trabalhos de pesquisa sobre adaptatividade divulgados no endereço  \url{http://lta.poli.usp.br/lta}. Nesta monografia, dispositivos do tipo autômato adaptativo serão utilizados para a apresentação do modelo operacional proposto por Vega~\cite{vega:2008}.
Inicialmente, apresentam-se os conceitos básicos que se referem ao dispositivo subjacente dirigido por regras de transição de estado.

Do ponto de vista conceitual, entende-se que o dispositivo subjacente ao autômato adaptativo, seja constituído por um conjunto de regras que mapeiam um par estado corrente/símbolo de entrada em um novo estado --- o estado seguinte. O diagrama de classes conceitual apresentado na Fig.~\ref{fig-regra-tradicional-uxf}, destaca os importantes elementos desta estrutura, a qual segue uma linha tradicional de formalização. Uma instância da classe \uml{Autômato Subjacente} é constituída por um conjunto de \uml{estados}, um dos quais é dito \uml{inicial}. Também parte desta estrutura é o conjunto dos estados \uml{finais}; um subconjunto de \uml{estados}. Durante a operação do autômato, um dos seus \uml{estados} é conhecido por estado \uml{corrente}. Além de estados, autômatos desta classe também são constituídos por um \uml{Espaço de Regras} --- um conjunto de \uml{regras}, cada uma delas modelada por uma instância de \uml{Regra Subjacente}. Esta parte do modelo de um \uml{Autômato Subjacente} suporta a transição de estados ao longo do reconhecimento de uma cadeia de símbolos~\cite{ramos:2009:lftmi}. Em particular, seja \uml{d} uma instância da classe  \uml{Autômato Subjacente}, com a regra \uml{r = (p, a, q)} $\in$ \uml{regras}. Se o estado \uml{d.corrente} for igual à \uml{r.p} e o estímulo de entrada corresponder à \uml{r.a}, então \uml{r.q} será o novo estado corrente do dispositivo: \uml{d.corrente = r.q}.

	\figLatHere[0.67]{\itaPastaFig}{fig-regra-tradicional-uxf}{Modelo conceitual de um autômato subjacente com destaque para o seu espaço de regras subjacentes.}

Estes elementos conceituais já são suficientes para o projeto inicial da gramática da linguagem de descrição de unidades biomiméticas adaptativas (LUBA). Pelegrini~\cite{pelegrini:2009} apresenta diversos conceitos e tecnologia para o projeto de linguagens que suportam um estilo de programação baseado na noção de adaptatividade. LUBA não é uma linguagem que oferece os recursos necessários para o desenvolvimento de programas em geral. A sua concepção enfatiza apenas os recursos necessários para a exposição das ideias desta monografia: regras de transição e funções adaptativas. A especificação da gramática LUBA será feita com a ferramenta ANTLR4~\cite{parr:2013}, cujos principais elementos podem ser encontrados no Anexo~\ref{anexo:isv:antlr4}.

A primeira regra de produção da gramática LUBA define o não-terminal \naoTerminal{uba} com a seguinte estrutura. No nível léxico, a palavra-reservada \terminal{uba} é seguida pelo identificador da nova classe de autômatos e, entre um par de abre/fecha-chaves, lista-se uma sequência (possivelmente vazia) de declarações de regras de transição. Na linguagem ANTLR4, o não-terminal de cada regra de produção é separado da sua definição por um \terminal{:} e as regras de produção, propriamente ditas, são separadas por um \terminal{;}. Assim, a regra para se especificar uma nova classe de autômatos \terminal{uba} fica:

\begin{lstlisting}[style=antlr]
<<UbaParser::seção de declarações de regras>>=
luba : UBA idClasse ABBL declaracao* FEBL ;
declaracao : decRegra | decFuncao | decMetodo ;
idClasse : IDENTIFICADOR ;
\end{lstlisting}

Posteriormente a linguagem será estendida para acomodar a declaração de funções adaptativas. A especificação de um autômato subjacente não envolve a chamada de funções adaptativas; é suficiente que se especifique qual o par estado corrente/símbolo de entrada dispara a ocorrência da transição para o estado seguinte. Como parte do projeto da linguagem LUBA (e visando facilitar um pouco a escrita das especificações), o estado inicial deverá ser prefixado com \lstinline[style=antlr]'>', enquanto que o prefixo para os estados finais será \lstinline[style=antlr]'!'. Assumindo-se estas convenções, escreve-se a produção do não terminal \lstinline[style=antlr]!decRegra! da seguinte forma:

\begin{lstlisting}[style=antlr]
<<UbaParser::declaração de regra adaptativa>>=
decRegra :
	cfa?
	ABPAR decEstOrg VIRGULA idSimbolo? FEPAR
	SETA decEstDst
	cfp?
	;
decEstOrg : INICIAL? idEstado ;
decEstDst : FINAL? idEstado ;
\end{lstlisting}

\noindent
Observa-se que a produção para o não-terminal \lstinline[style=antlr]!decRegra! admite a escrita de regras de transição em vazio: pode-se omitir o símbolo em uma transição entre dois estados.

Como um primeiro exemplo, a especificação do modelo de um autômato subjacente nos moldes daquele sugerido pelo diagrama mostrado na Fig~\ref{fig-regra-tradicional-uxf} tem a forma:

\begin{lstlisting}
<<exemplo1.uba>>=
uba Exemplo1 {
	( >q1, a ) -> q1
	( q1, b ) -> q2
	( q2, c ) -> !q3
}
\end{lstlisting}

\noindent
A criação de uma instância \uba{a1} da classe \uba{Exemplo1} é escrita da seguinte forma:

\begin{lstlisting}
<<apl1exemplo::instanciação de uma Uba>>=
def d1 = new cenario.Exemplo1()
\end{lstlisting}

\noindent
Na Fig~\ref{fig-cenario-Exemplo1-0-delta.png} encontra-se uma representação visual da topologia do autômato imediatamenta após a sua instanciação. O estado inicial é \estado{q1} (indicado por uma seta de entrada de estado) e, a partir dele, encontra-se uma auto-transição rotulada com o símbolo \simbolo{a}. Além dela, outra transição conduz ao estado \estado{q2}, um estado interno, na ocorrência de um símbolo \simbolo{b} na cadeia de entrada. O estado \estado{q3} é final e encontra-se conectado ao estado \estado{q2} por uma transição que depende do simbolo \simbolo{c} para ocorrer. Neste mesmo diagrama, o estado \estado{q1} encontra-se destacado: ele é o estado corrente do autômato recém instanciado. Instâncias do tipo \lstinline!Exemplo1! reconhecem as cadeias da linguagem denotada pela expressão regular $\mathbf{a^*bc}$, considerando-se o alfabeto $\Sigma=\{a, b, c\}$.

	\figLatHere[0.35]{\itaPastaFig}{fig-cenario-Exemplo1-0-delta.png}{Diagrama de estados de uma instância de \lstinline{Exemplo1} antes de entrar em operação. Em particular, o estado \estado{q1} é inicial e corrente, enquanto o estado \estado{q2} é interno e \estado{q3}, final.}

Do ponto de vista do processador da linguagem LUBA (a ser detalhado no Cap~\ref{cap-isv-execucao}), as regras de produção apresentadas até este momento fazem-no operar de modo que os seguintes efeitos sejam produzidos. Na parte correspondente à declaração da regra de transição não adaptativa \lstinline'( >q1, a ) -> !q1', reconhece-se a introdução de um novo estado \lstinline|q1| na forma \lstinline|>q1| (por ser a primeira ocorrência deste identificador em uma posição de \textit{estado} da regra de transição). O operador \lstinline|>| torna-o inicial na instância \lstinline|a1|. Assim, três efeitos devem ser produzidos:

   \begin{enumerate}

   \item declaração de um novo estado identificado por \lstinline|q1|;

   \item indicação que se trata do estado inicial de \lstinline|a1|;

   \item indicação que se trata do primeiro estado corrente quando tem início o processo de reconhecimento de uma cadeia de símbolos.

   \end{enumerate}

Nesta mesma regra de transição, observamos a forma \lstinline|!q1|. O operador \lstinline|!| torna o estado identificado por \lstinline|q1| um dos estados finais de \lstinline|a1|. Como não é a primeira ocorrência de \lstinline|q1| na descrição \lstinline!Exemplo1!, então não ocorrerá a declaração de um novo estado na instância \lstinline!a1!: aquele anteriormente vinculado a este identificador, além de ser inicial, também será um dos estados finais de \lstinline|a1|.

Agora, qual significado terá à ocorrência do símbolo \lstinline|a|? Corresponde a algum símbolo da cadeia de entrada a ser processado pelo autômato \lstinline|a1|. Ou seja, se o estado corrente de \lstinline|a1| corresponder ao identificador \lstinline|q1| e o símbolo a ser processado corresponder ao símbolo \lstinline|a|, então o autômato \lstinline|d1| passará para o estado identificado por \lstinline|q1|. Em uma notação inspirada na proposta apresentada por Neto e Bravo~\cite{neto:bravo:2002} e Dizeró~\cite{dizero:2010}, denota-se a classe de autômatos adaptativos \lstinline|Exemplo1| por $\mathit{Exemplo1}=(Q, \Sigma, q_0, F, \delta)$ sendo:

	\begin{itemize}
	\item $Q$ um conjunto finito de estados;
	\item $\Sigma$ um conjunto finito de estímulos de entrada;
	\item $q_0 \in Q$ o estado inicial;
	\item $F\subseteq Q$ o conjunto de estados finais;
	\item $\delta:Q\times\Sigma \rightarrow Q$ a função de transição de estados;
	\end{itemize}

\noindent e

	\begin{align*}
	Q = \{q_1, q_2, q_3\},\ \Sigma = \{a\},\ q_0 = q_1,\ F = \{ q_3\} \\
	\\
	\delta \triangleq \{ (q_1, a) \mapsto q_1, (q_1, b) \mapsto q_2, (q_2, c) \mapsto q_3 \} \tag{i}\label{d1-transicao}
	\end{align*}

\noindent
No caso do $\mathit{Exemplo1}$, a função de transição $\delta$ contém três pares de elementos, cada par correspondendo a uma regra de transição. A linguagem reconhecida por autômatos da classe \lstinline!Exemplo1! contém todas as cadeias com a forma especificada pela expressão $\mathbf{a^*bc}$.

A proxima seção introduz o suporte à descrição de funções adaptativas na linguagem LUBA.


