%\motto{Use the template \emph{chapter.tex} to style the various elements of your chapter content.}
\chapter{Adaptatividade e Semântica Operacional}
\label{cap-adaptatividade} % Always give a unique label
% use \chaptermark{}
% to alter or adjust the chapter heading in the running head

%\epigraph{The data are accumulating and the computers are humming, what we are lacking are the words, the grammar and the syntax of a new language...}{Dennis Bray (TIBS22(9):325-326,1997)}

% adaptatividade como problema computacional
% adaptatividade sob o ponto de vista biológico
% como estabelecer esta conexão?
% "Isso não cobre, no entanto, as necessidades daqueles usuários cujos programas necessitem efetuar alterações dinâmicas,  durante a execução, na lógica de seus procedimentos.  Para isso, tais programas necessitariam incorporar recursos de adaptatividade  e, portanto, dispor de meios para, sem intervenção externa, determinar e realizar alguma automodificação estrutural necessária." (JJNeto, Um Levantamento da Evolução da  Adaptatividade e da Tecnologia Adaptativa , 2007)
% IDEIA de conexão: o MTS produz uma especificação de modelo que embute a adaptatividade intrinsica ao comportamento biológico. Uma implementação "tradicional" de tal modelo, envolveria uma ação externa para ser implementado. A tecnologia adaptativa oferece recursos para implementar este modelo sem tal restrição.
% link para usar UML como notação de representação de modelos

\abstract{Neste capítulo, apresenta-se uma semântica operacional para autômatos adaptativos com base em regras de transição. Inicialmente, desenvolve-se o modelo semântico para suportar a especificação de instâncias de autômatos subjacentes. Em seguida, extende-se o modelo para acomodar o mecanismo adaptativo. Ao longo da apresentação, ilustra-se a utilização da LUBA em um exemplo de reconhecimento de cadeias de uma linguagem dependente de contexto.}

