\chapter{Cadeia de vacância: UBA inspirada em comportamento de espécie (UBA-CV)}

Em economia, cadeia de vacância é o nome dado ao conjunto de trocas sequenciais que beneficiam vários indivíduos sucessivamente. Como exemplo, no mercado imobiliário, a aquisição de uma casa nova por certo indivíduo propagará uma série de trocas secundárias que tendem a melhorar a situação de moradia de outros indivíduos. Os recursos que são afetados por esse fenômeno apresentam certas características em comum: devem ser desejados e relativamente difíceis de serem conseguidos, só podem ser ocupados por um único indivíduo de cada vez e só podem ser ocupados se estiverem vazios~\cite{chase12}.

Este tipo de fenômeno é também percebido no comportamento de alguns animais, dentre os quais, o \textit{Pagurus longicarpus}, uma espécie de caranguejo comum na costa leste dos EUA. Esses animais utilizam conchas como abrigos e as carregam consigo. Conforme crescem, os paguros procuram abrigos maiores e melhores abandonando suas antigas conchas que são utilizadas por animais mais novos e menores desencadeando uma série de trocas vantajosas para todos~\cite{chase12}.

Descobriu-se recentemente que os paguros usam dois tipos de cadeia. Na cadeia assíncrona apenas um caranguejo por vez encontra uma nova concha, na cadeia síncrona vários animais fazem fila, por ordem de tamanho, atrás do indivíduo que estiver examinando a concha vazia. Neste segundo caso, quando o primeiro caranguejo da fila se acomoda numa nova concha, o próximo ocupa a concha abandonada~\cite{chase12}.

\section{Modelagem semiótica da UBA-CV: passos 1 ao 5 do MTS}

\subsection{Passo 1: análise preliminar}

O estudo da cadeia de vacância no campo biológico pode trazer inúmeros \textit{insights} para soluções de problemas humanos, desde o ajuste de oferta e demanda no mercado imobiliário até a simples organização de filas de atendimento em corporações dos setores público e privado.

O comportamento adaptativo dos paguros fica mais claro na modalidade síncrona das trocas de conchas. Mediante a vacância percebida numa concha atraente (intacta e de bom tamanho), o que se esperaria de um animal de cérebro relativamente pequeno e simples seria a competição acirrada pelo novo abrigo. No entanto, o que se observa é um comportamento orquestrado, sugerindo a presença de cognição social sofisticada.

No caso dos paguros, existe um único fator (tamanho) responsável pela organização, mas nada impede de serem considerados conjuntos de fatores como organizadores da cadeia de vacância. Por exemplo, o atendimento preferencial em bancos e outras organizações, pode ser resolvido aplicando-se não apenas o critério de idade e outras características perceptíveis, mas também outros fatores não aparentes e que podem constar como informação relevante nos prontuários dos clientes.

\subsection{Passo 2: definição da arquitetura de subordinação}

No caso do comportamento assíncrono, apenas uma camada comportamental está presente: a busca e ocupação ocorrem de maneira serial. Mas, no comportamento síncrono, entre a busca e a ocupação, surge uma camada de espera em fila. Portanto, será considerado aqui o comportamento síncrono e suas duas camadas: busca/ocupação e posicionamento em fila.

\figLatHere[0.6]{\eduPastaFig}{ubacvsubordinacao-crop}{Arquitetura de subordinação abstraída do comportamento dos paguros.}

\subsection{Passo 3: definição do nível focal}

O nível focal adequado ao estudo da cadeia de vacância no comportamento dos paguros é o nível do organismo ou do indivíduo. Consequentemente, o nível inferior ou micro-semiótico, iniciador dos processos, é o nível dos processos neurais e o nível superior ou macro-semiótico, que apresenta as restrições naturais, é o ecológico %(Figura 3.2).

\figLatHere[0.7]{\eduPastaFig}{ubacvnivelfocal}{Representação dos níveis hierárquicos abstraídos do comportamento dos paguros.}

\subsection{Passo 4: levantamento das semioses relevantes}


Diferentemente dos casos anteriores, UBA-HS e UBA-TG,as pesquisas relativas à cadeia de vacância no comportamento dos paguros não leva em conta os processos internos, neurais ou celulares. O paguro seria, então, uma espécie de caixa preta e apenas seu comportamento como indivíduo é registrado. Assim, as representações decorrentes dos processos semióticos não serão especificadas, apenas consideradas como existentes.

Em resumo, no comportamento síncrono, o indivíduo se desloca em seu meio-ambiente e, ao detectar uma concha atraente abandonada, é impelido a uma troca, abandonando a concha atual e tomando para si o novo abrigo. No entanto, a presença de outros indivíduos faz com que o animal em questão se posicione em fila de acordo com seu tamanho~\cite{chase12}.

\subsubsection*{Camada de busca e ocupação}

Esta camada comportamental pode ser representada heuristicamente por três semioses relevantes: S1, S2 e S3. A semiose S1 seria a percepção da concha vazia, S2 a percepção da não existência (ou não existência) de concorrentes e S3 a ocupação da nova concha. Nota-se que , se o interpretante de S2 for a não existência de outros indivíduos, o percurso direto S1, S2 e S3 representaria uma troca assíncrona.

\begin{itemize}
	\item \textbf{Semiose 1 - Percepção nova concha.}

	Com o crescimento do animal, a concha em que habita se torna pequena demais. A percepção de uma nova concha provoca o início do processo da cadeia de vacância e o paguro se predispõe a ocupar da nova concha.

	Em termos semióticos (tríade Objeto/Signo/Interpretante) tem-se: a percepção de uma nova concha como o sinal perceptivo que designa o objeto (O), signo (S) não verificável e a predisposição para a ocupação como interpretante (I).

	\figLatHere[0.6]{\eduPastaFig}{ubacvs1-crop}{Características da semiose S1.}

	\item \textbf{Semiose 2 - Existência ou não de outros indivíduos.}

	O animal predisposto a ocupar a nova concha deve verificar se há outros indivíduos na mesma situação.

	Em termos semióticos (tríade Objeto/Signo/Interpretante) tem-se: predisposição à ocupação como o sinal perceptivo que designa o objeto (O), signo (S) não verificável e o resultado da verificação (há outros indivíduos ou não) como interpretante (I).

		\figLatHere[0.6]{\eduPastaFig}{ubacvs2-crop}{Características da semiose S2.}

	\item \textbf{Semiose 3 - Ocupação da nova concha.}

	Quando não existe outros indivíduos maiores do que o animal em questão, ocorre a ocupação

	Em termos semióticos (tríade Objeto/Signo/Interpretante) tem-se: Não existência de outros animais maiores como objeto (O), signo (S) não verificável e a efetivação da ocupação como interpretante (I).

			\figLatHere[0.6]{\eduPastaFig}{ubacvs3-crop}{Características da semiose S3.}

\end{itemize}

\subsubsection*{Camada de posicionamento em fila}

Esta camada comportamental é representada por duas semioses: S4 é responsável pela percepção dos diferentes tamanhos do indivíduos envolvidos e consequente posicionamento do animal em questão, e S5 é responsável pelo retorno à camada de ocupação quando o animal em questão percebe não haver mais nenhum indivíduo maior do que ele.

\begin{itemize}
	\item \textbf{Semiose 4 - Posicionamento em fila}

	A percepção de outros indivíduos no processo da cadeia de vacância faz o animal em questão se posicionar em fila segundo seu tamanho. Enquanto forem percebidos indivíduos maiores do que o animal em questão, a semiose S4 se repete.

	Em termos semióticos (tríade Objeto/Signo/Interpretante) tem-se: presença de outros indivíduos como sinal perceptivo que designa o objeto (O), signo (S) não verificável e o posicionamento em fila (I).

			\figLatHere[0.6]{\eduPastaFig}{ubacvs4-crop}{Características da semiose S4.}


	\item \textbf{Semiose 5 - Retorno à camada de busca e ocupação}

	Ao não perceber indivíduos maiores, o animal em questão, finalmente conclui a ocupação.

	Em termos semióticos (tríade Objeto/Signo/Interpretante) tem-se: não percepção de indivíduos maiores como objeto (O), signo (S) retorno à camada de busca e ocupação como interpretante (I).


			\figLatHere[0.6]{\eduPastaFig}{ubacvs5-crop}{Características da semiose S5.}

\end{itemize}

\subsection{Passo 5: modelagem semiótica}

Um animal específico tem a necessidade de encontrar uma nova concha mais adequada ao seu tamanho. Abaixo, segu a descrição com a omissão dos signos pois, como já declarado, não são verificáveis.

\subsubsection*{Camada de busca e ocupação}

Ao encontrar uma nova concha (objeto de S1), o paguro predispõe-se à troca (interpretante de S1). A predisposição (objeto de S2) provoca a verificação da presença ou ausência de outros indivíduos (interpretante de S2). Se não houver outros indivíduos (objeto de S3), a ocupação ocorre (interpretante de S3). Se houver outros indivíduos, o sistema altera para a camada de posicionamento em fila provocando S4.

\subsubsection*{Camada de posicionamento em fila}

A presença de outros indivíduos (objeto de S4) faz o animal em questão postar-se em fila por ordem de tamanho (interpretante de S4). Enquanto houver indivíduos maiores, S4 se repete. Ao verificar a não existência de outros indivíduos (objeto de S5) o sistema retorna à camada de ocupação (interpretante de S5)

\figLatTop[0.8]{\eduPastaFig}{ubacvmodelosemiotico-crop}{Diagrama representando o modelo semiótico abstraído da cadeia de vacância do comportamento do \textit{Pagurus longicarpus}.}

\section{Do modelo semiótico à UBA-CV: passo 6 do MTS}

Codificação da UBA-CV usando a linguagem de especificação UBA.

\section{Aplicação (didática): atendimento médico com prioridade }

\subsection{O problema}

Descrever o problema.

\subsection{Aplicação}
