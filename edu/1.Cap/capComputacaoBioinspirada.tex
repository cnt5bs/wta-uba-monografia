\chapter{Computação Bioinspirada}

% Ideia: sugerir que a inspiração na natureza ajuda a lidar com fenômenos ambientais.

O termo biomimética, tratado aqui como sinônimo de biomimese, biomimetismo e biônica, refere-se ao uso da natureza como inspiração para a busca de soluções para problemas humanos. Sua primeira aparição em um dicionário, \textit{Webster's Dictionary}, ocorreu em 1974 definindo-o como:  o estudo da formação, estrutura, ou função de substâncias e materiais biologicamente produzidos, bem como, o estudo de mecanismos e estruturas biológicas com o propósito de sintetizar produtos artificiais que os mimetizem. Assim, a biomimética é um campo de estudo relativamente novo que aborda o uso prático de mecanismos e funções da ciência biológica na engenharia e em outras áreas\cite{vincent06} ou, como escreve Benyus\cite{benyus03}, a biomimética é uma nova ciência que estuda os modelos da natureza e depois os imita ou se inspira neles ou em seus processos para resolver problemas humanos.

As aplicações mais comuns da biomimética relacionam-se ao desenvolvimento de produtos inspirados em estruturas biológicas. Este é o caso do trem-bala de \textit{Shinkansen} da \textit{West Japan Railway} cuja locomotiva teve o desenho inspirado na forma do bico de uma espécie de martim-pescador. Como resultado desta abordagem biomimética, houve redução substancial do nível de ruído percebido pelos passageiros, economia de 15\% de energia e alcance de velocidades 10\% superiores~\cite{biomimicry01}.

No entanto, comportamentos biológicos também podem servir de inspiração para a solução de problemas humanos complexos. Aqui, os campos da ciência da computação, da robótica e da ciência cognitiva fornecem vários exemplos: computação natural\cite{decastro06}, biorrobótica\cite{webb01} e inteligência artificial\cite{mccarthy55}, entre outros, são termos interconectados que se referem a abordagens biomiméticas na medida em que se inspiram em processos desenvolvidos pela natureza para alcançar seus objetivos.

% Ideia: apresentar a noção de computacão natural

Particularmente, a Computação Natural é formada por três ramos: (1) \textit{computação com materiais naturais}, cujo objetivo é a pesquisa de novos materiais biológicos que possam substituir a computação via silício; (2) \textit{simulação e emulação da natureza através da computação}, que é um processo sintético de se criar padrões, formas, comportamentos e organismos que podem mimetizar fenômenos naturais aumentando o entendimento da natureza e produzindo \textit{insights} sobre modelos computacionais e (3) \textit{computação inspirada pela natureza} tendo por objetivo o uso da natureza como inspiração para o desenvolvimento de técnicas de solução de problemas\cite{decastro06}. A diferença entre o segundo e o terceiro ramo é de intenção. Embora ambos possam partir de um mesmo fenômeno biológico, o segundo tem a intenção de estudo enquanto o terceiro tem a intenção de resolução de um problema. Sugere-se que ambos os ramos tenham a sua origem no trabalho de Turing. Em \cite{turing36} ele usa a mente humana como inspiração para resolver o \textit{Entscheidungsproblem} (terceiro ramo da Computação Natural). Em \cite{turing52} seu objetivo é apresentar um teste para verificar se a mente pode ser simulada (segundo ramo da Computação Natural).

% Ideia: mostrar que a origem dos modelos computacionais teve inspiração biológica
% OBS: Não valeria aqui destacar que a mente é fator biológico se descartarmos o dualismo cartesiano e adotarmos uma visão monista?

Deixando de lado o dualismo cartesiano entre mente e matéria e adotando uma posição monista na qual a mente reflete os processos biológicos subjacentes, a computação pode ser considerada, em parte, como um exemplo de tecnologia bioinspirada. Quando Turing propôs a sua máquina em \cite{turing36}, ele buscava retratar o comportamento de um computador humano\footnote{Na década de 1930, antes de existirem computadores automáticos, existiam pessoas contratadas para realizar cálculos (cômputos). O nome do cargo ou função administrativa que essas pessoas exerciam recebia o nome de \emph{computador} \cite{}.} durante a realização de cálculos. Além disso, a lógica Booleana, um elemento essencial no modelo de computador de von~Neumman, foi delineada no livro intitulado \textit{An Investigation of the Laws of Thought} \textit{cf.}~\cite{boole}. Isto sugere que Boole também trabalhou de maneira bioinspirada tendo a mente como base. Outrossim, von~Neumman parece seguir nessa linha de \emph{inspiração} biológica ao propor uma unidade responsável pelos cálculos e uma memória~\cite{Newmann:1945:FDR:1102046}. Percebe-se que Turing, Boole e von~Neumman usaram a mente humana como \emph{inspiração} para desenvolverem seus modelos computacionais. Neste sentido, é possível afirmar-se que seus modelos computacionais foram \emph{inspirados} no comportamento da mente humana: na sua origem, a computação foi bioinspirada.

% Ideia: enfatizar que a elaboração dos modelos computacionais tem foco na solução de problemas.

Convém mais uma vez distinguir a computação como simulação (segundo ramo da Computação Natural) da computação como \textit{problem-solving} (terceiro ramo da Computação Natural). A Ciência Cognitiva, por exemplo, utiliza simulações computacionais com o objetivo de estudar a mente. O próprio Turing fez uso da simulação para estudar sistemas biológicos~\cite{turing52}. Por outro lado, a Ciência da Computação estuda técnicas de \textit{problem-solving}, tais como redes neurais artificiais; computação evolucionária; inteligência coletiva; sistemas imunológicos artificiais; \textit{etc}. É neste contexto que o trabalho aqui apresentado se insere --- no âmbito da ciência da computação. Padrões biológicos que possam ser aplicados à solução de problemas humanos é um dos principais resultados alcançados. 

% Ideia: mostrar que não é suficiente a bioinspiração no caso de desenvolvimento tecnológico

Convém ainda considerar mais um ponto. Entre os algoritmos considerados ``clássicos'', muitos deles também são bioinspirados caso se considere o comportamento humano meramente como comportamento biológico. Por exemplo, os algoritmos de ordenação como \textit{bubble-sort}, \textit{selection-sort}, \textit{insertion-sort} baseiam-se em estratégias utilizadas por pessoas para ordenarem uma sequência de elementos. No entanto, alguns algoritmos como \textit{merge-sort} e \textit{quick-sort} são mecanismos criados sem inspiração biológica. Isso sugere que embora um comportamento bioinspirado possa ser útil na solução de um problema não implica que seja sempre a melhor ou a única solução. 

% Ideia: apresentar a noção de computação bioinspirada

Desta forma, em síntese, entende-se aqui Computação Bioinspirada (CB) como um particular funcionamento mecânico que visa retratar um comportamento de interesse apresentado por sistemas biológicos. A CB é um ramo da Computação Natural. Neste sentido a CB refere-se ao terceiro de seus sub-ramos apresentados.  Isso significa que a CB, objeto deste estudo, não tem como objetivo a simulação exata de todos os aspectos orgânicos de um dado fenômeno, o que se busca é a representação em termos de programa dos aspectos essenciais do fenômeno biológico, cuja utilidade é o suporte a aplicações em engenharia, finanças e outras áreas, essencialmente, distintas da biologia.

Neste ponto, cabe ressaltar o caráter heurístico da Computação Natural e, consequentemente da CB: \begin{quote}\textit{A maioria das abordagens computacionais com as quais a computação natural lida são baseadas em versões muito simplificadas de mecanismos e processos presentes nos fenômenos naturais correspondentes. São várias as razões para tais simplificações e abstrações. Primeiro de tudo, muitas simplificações são necessárias para se realizar, de forma tratável, a computação com um grande número de entidades. Também, pode ser vantajoso destacar as características mínimas necessárias para que certos aspectos particulares de um sistema sejam reproduzidos e para que propriedades emergentes sejam observadas. O nível mais apropriado para a investigação e abstração depende da questão científica proposta, do tipo de problema que se deseja resolver e do fenômeno biológico que se deseje sintetizar. A computação natural normalmente integra biologia experimental e teórica, física e química, observações empíricas da natureza e muitas outras ciências (...) para atingir seus objetivos.}~\cite{decastro06}\end{quote}

%a programação é artificial

Desta forma, a Computação Bioinspirada não procura igualar biologia à computação. Entende, ao contrário, que são dois campos distintos. Os fenômenos biológicos evoluíram naturalmente e apresentam comportamentos que, em princípio, nada têm a ver com funções computacionais. Não obstante, o olhar computacional do programador pode abstrair artificialmente aspectos dos modelos biológicos que induzem a uma interpretação computacional. Assim, a Computação Bioinspirada é uma abordagem heurística que busca estudar fenômenos naturais específicos, extraindo apenas seus elementos essenciais e passíveis de tratamento através abstrações computacionais.   

O próximo capítulo apresenta o Método de Transposição Semiótica (MTS). Trata-se, justamente, de um conjunto de passos heurísticos capazes de auxiliar o programador da Computação Bioinspirada a abstrair os elementos essenciais e computáveis de um dado fenômeno biológico.

%\section{Padrões de Biomiméticos} mais resultado do trabalho. (Já mencionado no 6º parágrafo acima)


%|=====|===========|===============================================|
%Um terceiro ponto que sugere a natureza bioinspirada da computação é a correspondência de Curry-Howard \cite{}. Provou-se nessa correspondência que uma prova escrita segundo o paradigma da matemática intuicionista equivale à um programa escrito segundo as bases do paradigma funcional (cáculo-$\lambda$). A matemática intuicionista, por sua vez, se baseia no trabalho de Brouwer \cite{brouwer} sobre o pensamento matemático.
