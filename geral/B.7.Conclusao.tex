
%\motto{Use the template \emph{chapter.tex} to style the various elements of your chapter content.}
\chapter{Conclusão}
\label{cap:conclusao} % Always give a unique label
% use \chaptermark{}
% to alter or adjust the chapter heading in the running head

% A seção de conclusão de cada capítulo prevê um balanço dos resultados obtidos como produto final do desenvolvimento das técnicas propostas e exercitadas no capítulo. Tratando-se de uma área inovadora, são raras as situações em que o trabalho associado aos assuntos tratados no capítulo não apresente novidades. Assim, espera-se que na conclusão do capítulo seja feito um levantamento de todas as inovações, e também de todas as possíveis contribuições que o conteúdo do capítulo possa ter trazido para as áreas de interesse a que se referem. Ideias novas que possam ter surgido devem ser indicadas como temas para trabalhos futuros, ou para possíveis publicações adicionais a serem desenvolvidas. Ensaios que tenham produzido resultados negativos devem ser analisados, e para as avaliações negativas, devem ser apresentadas justificativas ou alternativas para a correção ou melhoria do seu desempenho. Muito útil é terminar a conclusão do capítulo com um quadro de análise dos pontos altos e baixos identificados, com indicações objetivas de formas de correção ou de melhoramentos propostos.
